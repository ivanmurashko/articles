%% -*- coding:utf-8 -*- 
\documentclass[14pt,a4paper]{article} 

\usepackage[utf8]{inputenc}
\usepackage{minted}
\usepackage{longtable}
\usepackage{hyperref}
\hypersetup{
  pdftex,
  allcolors=blue,
  bookmarksnumbered=true,     
  bookmarksopen=true,         
  bookmarksopenlevel=1,       
  colorlinks=true,            
  pdfstartview=Fit,           
  pdfpagemode=UseOutlines,  
  pdfpagelayout=TwoPageRight,
  pdftitle={STL sort},
  pdfsubject={STL sort},
  pdfauthor={Ivan Murashkо},
  pdfkeywords={STL, C++, quick sort}
}

\title{Inside STL sort algorithm}
\author{Murashko Ivan}
\date{}
\begin{document}

\maketitle

\section*{Introduction}
The most tricky thing in C++ is an error catch. Most part of errors
are located in the application level and you can be quite sure that
the library, especially the well known STL, is error free. But what
should you do if your application code is trivial i.e. seems to be
error free but you got a SIGSEGV inside a system library. Most
probably you just felt into a situation that is described in the
article. 

Note: All examples in the article uses gcc 5.5.0 from the docker image
gcc:5: 
\begin{minted}{shell}
$ docker run gcc:5 gcc --version
gcc (GCC) 5.5.0
\end{minted}


\section{Example}
Lets look at the following code:
\inputminted{c++}{./src/sort.cpp}
What output does it produces? One can expect the following one:
\begin{minted}{shell}
$ ./src/sort
1 1 1 1 1 1 1 1 1 1 1 1 1 1 1 1 1
$
\end{minted}
i.e. the original array has to be displayed. In reality the following
output will be produced by gcc 5.5.0 
\begin{minted}{shell}
$ ./src/sort
0 1 1 1 1 1 1 1 1 1 1 1 1 1 1 1 1
$
\end{minted}
As one can see the array is broken. How it can be if every
function in the code snapshot seems to be correct? The answer is
below. 

\section{Inside STL sort}

The STL library states the following requirements for std::sort 
\cite{ISO:2013:IIP}(pp. 897-898): the compare function must follow so
called strict weak ordering requirement. Especially it
says that 
\begin{minted}{c++}
comp(x, x) == true;
\end{minted}
The requirement is violated
by the $\le$ operand used in our example.  

Why the requirement is so important? The answer is in the algorithm
that is used for sorting. STL uses an optimized version of Quicksort algorithm
\cite{wiki:quicksort}. 

The code that produces the problem can be found via the following command
\begin{minted}{shell}
docker run gcc:5 tail -n +1888 \
/usr/local/include/c++/5.5.0/bits/stl_algo.h | head -n 21
\end{minted}
and looks as follow
\begin{minted}{c++}
  /// This is a helper function...
  template<typename _RandomAccessIterator, typename _Compare>
    _RandomAccessIterator
    __unguarded_partition(_RandomAccessIterator __first,
                          _RandomAccessIterator __last,
                          _RandomAccessIterator __pivot, _Compare __comp)
    {
      while (true)
        {
          while (__comp(__first, __pivot))
            ++__first;
          --__last;
          while (__comp(__pivot, __last))
            --__last;
          if (!(__first < __last))
            return __first;
          std::iter_swap(__first, __last);
          ++__first;
        }
    }
\end{minted}
As soon as our example have all elements equal to 1 one can assume
that \textbf{\_\_pivot} also keeps $1$. In the function we start to
compare all elements with the pivot. The used comparator will always
produce \textbf{true} as soon 
as $1 \le 1$. The size of the array is 17 
\begin{minted}{shell}
(gdb) p data.size()
$1 = 17
\end{minted}
and when we pass through whole array and reach its end we
will found $0$ on the last position (\textbf{\_\_last}): 
\begin{minted}{shell}
(gdb) p *(&data[0] + 17)
$2 = 0
\end{minted}
The found element, that is located outside the array, will be
included into the search as soon as it also satisfied the required
property: $0 \le 1$. The next element will be greater  
\begin{minted}{shell}
(gdb) p *(&data[0] + 18)
$3 = 61777
\end{minted}
and therefore the \textbf{\_\_first} iterator will be 1 position
behind the original \textbf{\_\_last} iterator. The value will be
returned into the following helper function
\begin{minted}{c++}
  /// This is a helper function...
  template<typename _RandomAccessIterator, typename _Compare>
    inline _RandomAccessIterator
    __unguarded_partition_pivot(_RandomAccessIterator __first,
				_RandomAccessIterator __last, _Compare __comp)
    {
      _RandomAccessIterator __mid = __first + (__last - __first) / 2;
      std::__move_median_to_first(__first, __first + 1, __mid, __last - 1,
				  __comp);
      return std::__unguarded_partition(__first + 1, __last, __first, __comp);
    }
\end{minted}
that will return the result as \textbf{\_\_cut} in 
\begin{minted}{c++}
  /// This is a helper function for the sort routine.
  template<typename _RandomAccessIterator, typename _Size, typename _Compare>
    void
    __introsort_loop(_RandomAccessIterator __first,
		     _RandomAccessIterator __last,
		     _Size __depth_limit, _Compare __comp)
    {
      while (__last - __first > int(_S_threshold))
	{
	  if (__depth_limit == 0)
	    {
	      std::__partial_sort(__first, __last, __last, __comp);
	      return;
	    }
	  --__depth_limit;
	  _RandomAccessIterator __cut =
	    std::__unguarded_partition_pivot(__first, __last, __comp);
	  std::__introsort_loop(__cut, __last, __depth_limit, __comp);
	  __last = __cut;
	}
    }
\end{minted}
The \textbf{\_\_last} iterator will be replaced with \textbf{\_\_cut}
that is \textbf{\_\_first + 18} or \textbf{\_\_last + 1}. Thus 
\textbf{std::\_\_partial\_sort} will get the range for sort that is
greater (by 1) then original one. As result the element ($0$) outside
array will be placed into beginning for the vector. 

\section*{Conclusion}
As you can see, the usage comparator that violates strict weak ordering
requirements will lead to array boundary condition violations. In our
example we were ``happy'' and just got incorrect data in the result. The
SIGSEGV is more probable result for such error. The error can be easy
fixed with using correct comparator, for instance the following one
\begin{minted}{c++}
  auto comp = [](int i1, int i2) { return i1 < i2; };
\end{minted}

\bibliographystyle{gost780s}  
\bibliography{cppsort}     


\end{document}

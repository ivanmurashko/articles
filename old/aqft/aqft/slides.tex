\documentclass{beamer}
\usepackage[utf8]{inputenc}
\usepackage[english]{babel}
\usepackage{amsmath}
\usepackage{amsthm}

\begin{document}

\begin{frame}
  \frametitle{Analyze of Quantum Fourier Transform Circuit Implementation}
  \framesubtitle{Ivan Murashko, Constantine Korikov. Peter the Great
    St.Petersburg Polytechnic University, St. Petersburg, Russia}

  Quantum Fourier transform circuit is the key element of quantum
  computation. Originally it was introduced in the Shor's paper
  that also contains a proof of correctness for it. There
  is another proof of correctness that shows the role of each element
  of the circuit and can be used for the scheme analyze. 

\end{frame}

\begin{frame}
  \frametitle{Fourier transform}  
  \begin{eqnarray}
    \left\{x_m\right\} \rightarrow \left\{\tilde{X}_k\right\}
    \nonumber \\
    \tilde{X}_k = \sum^{M - 1}_{m = 0} x_m e^{-i \frac{2 \pi}{M} k\cdot m}
    \nonumber
  \end{eqnarray}
\end{frame}

\begin{frame}
  \frametitle{Quantum Fourier Transform}
  \begin{eqnarray}
    \left|x\right> = \sum_{k = 0}^{M - 1}x_k \left|k\right>
    \nonumber \\
    \left|\tilde{X}\right>_{inv} = \sum_{j = 0}^{M - 1}
    \tilde{X}_j\left|j\right>_{inv}
    \nonumber
  \end{eqnarray}
  \begin{eqnarray}
    \hat{F}^{M}\left|k\right> = \frac{1}{\sqrt{M}}\sum_{j = 0}^{M -1}
    e^{-i \frac{2 \pi}{M} k j}\left|j\right>_{inv}
    \nonumber \\
    \hat{F}^{M} \left|x\right> = \left|\tilde{X}\right>_{inv},
    \nonumber
  \end{eqnarray}
\end{frame}

\begin{frame}
  \frametitle{Input data}
  \input ./figquantfourier0.tex
\end{frame}

\begin{frame}
  \frametitle{Step 1}
  \input ./figquantfourier1.tex
\end{frame}

\begin{frame}
  \frametitle{Step 2}
  \input ./figquantfourier2.tex
\end{frame}

\begin{frame}
  \frametitle{Final circuit}
  \input ./figquantfourier3.tex
\end{frame}


\end{document}

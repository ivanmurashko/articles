%% -*- coding:utf-8 -*- 
\documentclass{llncs}

\begin{document}

\title{Analyze of Quantum Fourier Transform Circuit Implementation}
\author{Ivan Murashko, Constantine Korikov}
\institute{Peter the Great St. Petersburg Polytechnic University,
  St. Petersburg, 195251. Russia,\\
  \email{ivan.murashko@gmail.com},
  \email{korikov.constantine@spbstu.ru}
}
\maketitle

\begin{abstract}
  Quantum Fourier transform circuit is the key element of quantum computation.
  Originally it was introduced in the Shor's paper containing a classical
  proof of correctness. We propose another corroboration that shows
  the role of each element of the circuit and can be used for the scheme analyze.
  \keywords{
    quantum Fourier transform,
    Shor's algorithm,
    quantum computing 
  }
\end{abstract}

\section{Introduction}
Discrete Fourier transform (DFT) can be used for periodic sequence
(functions) analysis. The transformation is defined by the following
equation: 
\begin{equation}
\tilde{X}_k = \sum^{M - 1}_{m = 0} x_m e^{-i \frac{2 \pi}{M} k\cdot m},
\label{eqPart4QuantCompShorFourierDiscretFourier}
\end{equation}
where the input sequence $\left\{x_m\right\}$ has $M$
members.

Shor \cite{shor} suggested a quantum circuit
(see Fig. \ref{figQuantCompQuantFourier}d) 
that can be used for DFT
in quantum algorithms \cite{shor,kitaev,lanyon,ettinger} especially in the
well known Shor's algorithm for factorization problem solving
\cite{shor}.  

Quantum Fourier transform \cite{shor} operates with the following states
\begin{equation}
\left|x\right> = \sum_{k = 0}^{M - 1}x_k \left|k\right>,
\label{eqPart4QuantCompShorQuantFourierSeries}
\end{equation}
where the sequence of amplitudes $\left\{x_k\right\}$ forms initial
sequence for the Fourier transform
(\ref{eqPart4QuantCompShorFourierDiscretFourier}). 
The basis vector $\left|k\right>$ keeps the number of a sequence's element.   

The state (\ref{eqPart4QuantCompShorQuantFourierSeries})
should be normalized:
\[
\sum_k\left|x_k\right|^2 = 1.
\]

Lets an operator $\hat{F}^{M}$ (quantum Fourier operator) transforms
basis vector  $\left|k\right>$ by means of a rule defined by equation 
(\ref{eqPart4QuantCompShorFourierDiscretFourier}):
\begin{equation}
\hat{F}^{M}\left|k\right> = \frac{1}{\sqrt{M}}\sum_{j = 0}^{M -1}
e^{-i \omega k j}\left|j\right>_{inv}, 
\label{eqPart4QuantCompShorQuantFourierBasis}
\end{equation}
where $\omega = \frac{2 \pi}{M}$.
Basis vectors $\left\{\left|k\right>\right\}$ and
$\left\{\left|k\right>_{inv}\right\}$ are the same set of vectors
which are enumerated in a different order
(see Fig. \ref{figQuantCompQuantFourier}a and Example
\ref{exInputData}). 

From (\ref{eqPart4QuantCompShorQuantFourierSeries}) and
(\ref{eqPart4QuantCompShorQuantFourierBasis}) one can get
\begin{eqnarray}
\hat{F}^{M}\left|x\right> = \sum_{j = 0}^{M - 1}x_k \hat{F}^{M}
\left|k\right> = 
\nonumber \\
= \frac{1}{\sqrt{M}}\sum_{k = 0}^{M -1}\sum_{j = 0}^{M - 1}
e^{-i \omega k j}x_k\left|j\right>_{inv} = 
\nonumber \\
= \sum_{j = 0}^{M - 1} \left\{\frac{1}{\sqrt{M}}\left(
\sum_{k = 0}^{M - 1}e^{-i \omega k j} x_k
\right)\right\}\left|j\right>_{inv} = 
\nonumber \\
= \sum_{j = 0}^{M - 1}\tilde{X}_j\left|j\right>_{inv} = \left|\tilde{X}\right>_{inv},
\label{eqPart4QuantCompShorQuantFourierEnd1}
\end{eqnarray}
where
\begin{equation}
\tilde{X}_j = \tilde{X}_j^{M} = 
\frac{1}{\sqrt{M}}\sum_{k = 0}^{M - 1}e^{-i \omega k j} x_k.
\label{eqPart4QuantCompShorQuantFourierEnd}
\end{equation}
The equation (\ref{eqPart4QuantCompShorQuantFourierEnd}) conforms the
classical ones that is used for discrete Fourier transform
(\ref{eqPart4QuantCompShorFourierDiscretFourier}), i.e. one can write
\[
 \left|x\right> \longleftrightarrow \left|\tilde{X}\right>_{inv}.
\]

We are going to show that the circuit shown on
Fig. \ref{figQuantCompQuantFourier}d does the transformation
defined by (\ref{eqPart4QuantCompShorQuantFourierEnd}). The subject of
analyze consists of several elements: quantum Fourier transform gate for higher
bits ($\hat{F}^{\frac{M}{2}}$), phase shift and Hadamard
gate (see Fig. \ref{figQuantCompQuantFourier}). If the elements
are added to the circuit step by step then the different
transformations of initial data are performed. The transformations are
described below in details.

\section{Input data}

Lets the input data for the system is a quantum state 
(\ref{eqPart4QuantCompShorQuantFourierSeries}) that is a superposition
of $M$ basis states $\left\{\left|k\right>\right\}$ 
(see Fig. \ref{figQuantCompQuantFourier}a). 
Lets the number of basis states is a power of 2, i.e. a basis state
can be present as a tensor product of 
$n = \log_2{M}$ qbits:
\begin{equation}
\left|k\right> = \left|a^{(k)}_0\right> \otimes  \left|a^{(k)}_1\right>
\otimes \cdots \otimes \left|a^{(k)}_{n-1}\right>, 
\nonumber
\end{equation}
where
\begin{eqnarray}
k = a^{(k)}_0 + 2^1 a^{(k)}_1 + \dots + 2^{n-1} a^{(k)}_{n-1},
\nonumber \\
a^{(k)}_i \in \left\{0, 1\right\}.
\nonumber
\end{eqnarray}

There is a superposition of $M$ basis states 
$\left\{\left|j\right>_{inv}\right\}$ on the input
(see Fig. \ref{figQuantCompQuantFourier}a).
For the state
$\left|j\right>_{inv}$ one can get 
\begin{equation}
\left|j\right>_{inv} = \left|b^{(j)}_{n-1}\right> \otimes
\left|b^{(j)}_{n-2}\right> 
\otimes \cdots \otimes \left|b^{(j)}_{0}\right>, 
\nonumber
\end{equation}
where
\begin{eqnarray}
j = b^{(j)}_0 + 2^1 b^{(j)}_1 + \dots + 2^{n-1} b^{(j)}_{n-1},
\nonumber \\
b^{(j)}_i \in \left\{0, 1\right\}.
\nonumber
\end{eqnarray}

\begin{example}[Input data]
  \label{exInputData}
  Let $M=8$ i.e. $n=3$. In the case the number 3 is represented as
  follows
  \[
  \left|3\right> = \left|1\right> \otimes  \left|1\right>
  \otimes  \left|0\right>.
  \]
  and
  \[
  \left|3\right>_{inv} = \left|0\right> \otimes  \left|1\right>
  \otimes  \left|1\right>.
  \]
\end{example}


Using well known Cooley–Tukey fast Fourier transform algorithm
\cite{fft} the (\ref{eqPart4QuantCompShorFourierDiscretFourier}) can be
rewritten in the following form
\begin{equation}
\tilde{X}_k = \sum^{\frac{M}{2} - 1}_{m = 0} F_{k,m}^{\frac{M}{2}} x_{2m} +
\exp{\left(-2\pi i \frac{k}{M}\right)}
\sum^{\frac{M}{2} - 1}_{m = 0}  F_{k,m}^{\frac{M}{2}} x_{2m + 1}.
\label{eqAddFourierDiscretFourierFFT}
\end{equation}
We can see (\ref{eqAddFourierDiscretFourierFFT}) that if an
input signal $x$ consists of $n = \log_2{M}$ 
bits when bit $a^{(k)}_0$ can be used to choose even 
(the first part of sum (\ref{eqAddFourierDiscretFourierFFT}))
or odd
(the second part of sum (\ref{eqAddFourierDiscretFourierFFT}))
members. 

Thus the state
(\ref{eqPart4QuantCompShorQuantFourierSeries}) 
can be presented in the form of sum of even and odd components: 
\begin{eqnarray}
\left|x\right> = \sum_{k = 0}^{M - 1}x_k \left|k\right> = 
\sum_{k = 0}^{M - 1}x_k \left|a^{(k)}_0\right> \otimes  \left|a^{(k)}_1\right>
\otimes \cdots \otimes \left|a^{(k)}_{n-1}\right> = 
\nonumber \\
 = \sum_{m = 0}^{\frac{M}{2} - 1}x_{k=2m} \left|0\right> \otimes  \left|a^{(k)}_1\right>
\otimes \cdots \otimes \left|a^{(k)}_{n-1}\right> +
\nonumber \\
+
\sum_{m = 0}^{\frac{M}{2} - 1}x_{k=2m + 1} \left|1\right> \otimes  \left|a^{(k)}_1\right>
\otimes \cdots \otimes \left|a^{(k)}_{n-1}\right> = 
\nonumber \\
 = \sum_{m = 0}^{\frac{M}{2} - 1}x_{k=2m} \left|0\right> \otimes  \left|m\right> +
\sum_{m = 0}^{\frac{M}{2} - 1}x_{k=2m + 1} \left|1\right> \otimes  \left|m\right> = 
\nonumber \\
= \sum_{m = 0}^{\frac{M}{2} - 1}x_{2m} \left|2m\right> +
\sum_{m = 0}^{\frac{M}{2} - 1}x_{2m + 1} \left|2m+1\right>,
\nonumber
\end{eqnarray}
where
\begin{equation}
m = a^{(k)}_1 + 2^1 a^{(k)}_2 + \dots + 2^{n-2} a^{(k)}_{n-1}.
\nonumber
\end{equation}

\section{Fourier transform for high bits}

With Fourier transform for high bits only in $\hat{F}^{\frac{M}{2}}$,
i.e. with excluding $a^{(k)}_0$ we can obtain (see
Fig. \ref{figQuantCompQuantFourier}b): 
\begin{eqnarray}
\left|x\right> \rightarrow
\hat{F}^{\frac{M}{2}} \sum_{m = 0}^{\frac{M}{2} - 1}x_{2m} \left|2m\right> +
\hat{F}^{\frac{M}{2}} \sum_{m = 0}^{\frac{M}{2} - 1}x_{2m + 1}
\left|2m+1\right> = 
\nonumber \\
=
\hat{F}^{\frac{M}{2}} \sum_{m = 0}^{\frac{M}{2} - 1}x_{2m} 
\left|0\right> \otimes  \left|m\right> +
\hat{F}^{\frac{M}{2}} \sum_{m = 0}^{\frac{M}{2} - 1}x_{2m + 1}
\left|1\right> \otimes  \left|m\right>
=
\nonumber \\
=
\sum_{m = 0}^{\frac{M}{2} - 1}x_{2m} 
\left|0\right> \otimes \hat{F}^{\frac{M}{2}} \left|m\right> +
\sum_{m = 0}^{\frac{M}{2} - 1}x_{2m + 1}
\left|1\right> \otimes \hat{F}^{\frac{M}{2}} \left|m\right>.
\label{eqPart4QuantCompShorQuantFourierStep1}
\end{eqnarray}
Using (\ref{eqPart4QuantCompShorQuantFourierBasis}) it can be got
\begin{equation}
\hat{F}^{\frac{M}{2}} \left|m\right> = \sqrt{\frac{2}{M}}
\sum_{j= 0}^{\frac{M}{2} - 1} e^{-i \frac{4 \pi}{M} m j}\left|j\right>_{inv}.
\nonumber
\end{equation}
Thus for (\ref{eqPart4QuantCompShorQuantFourierStep1}) it holds
\begin{eqnarray}
\left|x\right> \rightarrow
\sum_{m = 0}^{\frac{M}{2} - 1}x_{2m} 
\left|0\right> \otimes \hat{F}^{\frac{M}{2}} \left|m\right> +
\sum_{m = 0}^{\frac{M}{2} - 1}x_{2m + 1}
\left|1\right> \otimes \hat{F}^{\frac{M}{2}} \left|m\right> = 
\nonumber \\
=
\sqrt{\frac{2}{M}} \sum_{j = 0}^{\frac{M}{2} - 1} e^{-i \frac{4 \pi}{M} m j} 
\sum_{m = 0}^{\frac{M}{2} - 1}x_{2m} \left|0\right> \otimes
\left|j\right>_{inv}
+
\nonumber \\
+
\sqrt{\frac{2}{M}} \sum_{j = 0}^{\frac{M}{2} - 1} e^{-i \frac{4 \pi}{M} m j} 
\sum_{m = 0}^{\frac{M}{2} - 1}x_{2m+1} \left|1\right> \otimes
\left|j\right>_{inv}
=
\nonumber \\
=
\sum_{j = 0}^{\frac{M}{2} - 1}  
\left( \sqrt{\frac{2}{M}} 
\sum_{m = 0}^{\frac{M}{2} - 1} e^{-i \frac{4 \pi}{M} m j} x_{2m} 
\right) \left|j\right>_{inv}
+
\nonumber \\
+
\sum_{j = 0}^{\frac{M}{2} - 1}
\left( \sqrt{\frac{2}{M}}  
\sum_{m = 0}^{\frac{M}{2} - 1}e^{-i \frac{4 \pi}{M} m j} x_{2m+1} 
\right)
\left|\frac{M}{2} + j\right>_{inv}
=
\nonumber \\
= \sum^{\frac{M}{2} - 1}_{j = 0}  \tilde{A}_{j} \left|j\right>_{inv} +
\sum^{\frac{M}{2} - 1}_{j = 0}  \tilde{B}_{j} \left|\frac{M}{2} + j\right>_{inv},
\label{eqStep1}
\end{eqnarray}
where
\begin{eqnarray}
\tilde{A}_{j} = 
\sqrt{\frac{2}{M}} 
\sum_{m = 0}^{\frac{M}{2} - 1} e^{-i \frac{4 \pi}{M} m j} x_{2m} 
\nonumber \\
\tilde{B}_{j} =
\sqrt{\frac{2}{M}} 
\sum_{m = 0}^{\frac{M}{2} - 1} e^{-i \frac{4 \pi}{M} m j} x_{2m+1} 
\label{eqPart4QuantCompShorAB}
\end{eqnarray}

\section{Phase shift}

As we can see the equations (\ref{eqStep1}, \ref{eqPart4QuantCompShorAB}) do
not correspond to the required Fourier transform
(\ref{eqPart4QuantCompShorQuantFourierEnd1}).
One of the problem is that components $x_{2m}$ and $x_{2m+1}$ in
(\ref{eqPart4QuantCompShorAB}) have the same phase but it should
differ. The problem can be fixed if a phase shift is added for odd
components i.e.  for elements with $a_0^k = 1$. As result the circuit 
shown on Fig. \ref{figQuantCompQuantFourier}c is got: 
\begin{eqnarray}
\left|x\right> \rightarrow
\hat{F}^{\frac{M}{2}} \sum_{m = 0}^{\frac{M}{2} - 1}x_{2m} \left|2m\right> +
\hat{R}\hat{F}^{\frac{M}{2}} \sum_{m = 0}^{\frac{M}{2} - 1} x_{2m + 1}
\left|2m+1\right> =
\nonumber \\
= 
\sum^{\frac{M}{2} - 1}_{j = 0} \tilde{A}_{j} \left|j\right>_{inv} +
\sum^{\frac{M}{2} - 1}_{j = 0}  
\tilde{B}_{j} \hat{R}\left|\frac{M}{2} + j\right>_{inv},
\nonumber \\
= 
\sum^{\frac{M}{2} - 1}_{j = 0}  \tilde{A}_{j} \left|j\right>_{inv} +
\sum^{\frac{M}{2} - 1}_{j = 0}  
\tilde{C}_{j} \left|\frac{M}{2} + j\right>_{inv}.
\label{eqPart4QuantCompShorFourierStep2}
\end{eqnarray}
With equation
\begin{equation}
\hat{R}_l \left|b^{(j)}_l\right> = 
exp{\left(- 2 \pi i \frac{b^{(j)}_l}{2^{n-l}}\right)}
\left|b^{(j)}_l\right>
\nonumber
\end{equation}
we can get that operator $\hat{R}$ applied to state  
$\left|\frac{M}{2} + j\right>_{inv}$ will produce the following result:
\begin{eqnarray}
\hat{R}\left|\frac{M}{2} + j\right>_{inv} = 
\hat{R}\left|1\right> \otimes  \left|j\right>_{inv} = 
\nonumber \\
= 
\left|1\right> \otimes \hat{R}_0 \left|b^{(j)}_0\right>
\otimes \cdots \otimes \hat{R}_{n-2} \left|b^{(j)}_{n-2}\right> = 
\nonumber \\
= 
\prod_{l = 0}^{n-2}exp{\left(- 2 \pi i \frac{2^lb^{(j)}_l}{2^n}\right)}
\left|1\right> \otimes \left|j\right>_{inv} = 
\nonumber \\
=
exp{\left(-2 \pi i \frac{j}{M}\right)}
\left|\frac{M}{2} + j\right>_{inv} 
\label{eqPart4QuantCompShorFourierStep2Pre}
\end{eqnarray}
We used the following fact
\[
j = b^{(j)}_0 + 2^1 b^{(j)}_1 + \dots + 2^{n-2} b^{(j)}_{n-2}
\]
for (\ref{eqPart4QuantCompShorFourierStep2Pre}) derivation.

Thus for $\tilde{C}_{j}$ in 
(\ref{eqPart4QuantCompShorFourierStep2}) it can be got
\begin{eqnarray}
\tilde{C}_{j} = 
\sqrt{\frac{2}{M}} 
\sum_{m = 0}^{\frac{M}{2} - 1} 
e^{- 2 \pi i \frac{j}{M}}
e^{-i \frac{4 \pi}{M} m j} x_{2m+1} =
\nonumber \\
=
\sqrt{\frac{2}{M}} 
\sum_{m = 0}^{\frac{M}{2} - 1} 
e^{-i \frac{2 \pi}{M} \left(2m+1\right) j} x_{2m+1}
\label{eqPart4QuantCompShorC}
\end{eqnarray}

\section{Hadamard transformation for the lowest bit}
The result after phase shift still does not produce the required
(Fourier) transform and additional manipulations are
required. Especially the result of sum for even components is that
required. Another situation with odd ones where sum should be done
with additional phase shift. The required transformation is provided
by Hadamard gate. 

If the Hadamard transformation is applied for the qbit 
$\left|a_0\right>$ then the circuit shown on 
Fig. \ref{figQuantCompQuantFourier}d is got. 
The initial state is transformed by the following law:
\begin{eqnarray}
\left|x\right> \rightarrow
\hat{H}\hat{F}^{\frac{M}{2}} \sum_{m = 0}^{\frac{M}{2} - 1}x_{2m} \left|2m\right> +
\hat{H}\hat{R}\hat{F}^{\frac{M}{2}}\sum_{m = 0}^{\frac{M}{2} - 1} x_{2m + 1} =
\nonumber \\
=
\sum_{j = 0}^{\frac{M}{2} - 1}
\tilde{A}_{j}
\hat{H}\left|0\right> \otimes \left|j\right>_{inv}
+
\sum_{j = 0}^{\frac{M}{2} - 1} 
\tilde{C}_{j}
\hat{H}\left|1\right> \otimes \left|j\right>_{inv} 
=
\nonumber \\
= 
\frac{1}{\sqrt{2}}\sum_{j = 0}^{\frac{M}{2} - 1}
\tilde{A}_{j} 
\left(\left|0\right> + \left|1\right> \right) \otimes  
\left|j\right>_{inv}
+
\frac{1}{\sqrt{2}}\sum_{j = 0}^{\frac{M}{2} - 1}
\tilde{C}_{j} 
\left(\left|0\right> - \left|1\right> \right) \otimes  
\left|j\right>_{inv}
=
\nonumber \\
=
\sum_{j = 0}^{\frac{M}{2} - 1}
\frac{\tilde{A}_{j} + \tilde{C}_{j} }{\sqrt{2}} 
\left|0\right> \otimes \left|j\right>_{inv} +
\sum_{j = 0}^{\frac{M}{2} - 1}
\frac{ \tilde{A}_{j} - \tilde{C}_{j}}{\sqrt{2}} 
\left|1\right> \otimes \left|j\right>_{inv}
=
\nonumber \\
=
\sum_{j = 0}^{\frac{M}{2} - 1}
\frac{\tilde{A}_{j} + \tilde{C}_{j} }{\sqrt{2}} \left|j\right>_{inv} +
\sum_{j = 0}^{\frac{M}{2} - 1}
\frac{ \tilde{A}_{j} - \tilde{C}_{j}}{\sqrt{2}} 
\left|\frac{M}{2} + j \right>_{inv}.
\label{eqPart4QuantCompShorFourierStep3}
\end{eqnarray}
For (\ref{eqPart4QuantCompShorFourierStep3}) members using 
(\ref{eqPart4QuantCompShorAB}) and (\ref{eqPart4QuantCompShorC}) one
can conclude:
\begin{eqnarray}
\frac{\tilde{A}_{j} + \tilde{C}_{j} }{\sqrt{2}} = 
\sqrt{\frac{1}{M}} 
\sum_{m = 0}^{\frac{M}{2} - 1} e^{-i \frac{4 \pi}{M} m j} x_{2m}  +
\sqrt{\frac{1}{M}} 
\sum_{m = 0}^{\frac{M}{2} - 1} 
e^{-i \frac{2 \pi}{M} \left(2m+1\right) j} x_{2m+1} = 
\nonumber \\
=
\sqrt{\frac{1}{M}} \sum_{m = 0}^{M - 1}
e^{-i \frac{2 \pi}{M} m j} x_{m}
\label{eqPart4QuantCompShorFourierStep3_1}
\end{eqnarray}
and
\begin{eqnarray}
\frac{\tilde{A}_{j} - \tilde{C}_{j} }{\sqrt{2}} = 
\sqrt{\frac{1}{M}} 
\sum_{m = 0}^{\frac{M}{2} - 1} e^{-i \frac{4 \pi}{M} m j} x_{2m}  -
\sqrt{\frac{1}{M}} 
\sum_{m = 0}^{\frac{M}{2} - 1} 
e^{-i \frac{2 \pi}{M} \left(2m+1\right) j} x_{2m+1}
= 
\nonumber \\
=
\sqrt{\frac{1}{M}} \sum_{m = 0}^{M - 1}
e^{-i \frac{2 \pi}{M} m j} x_{m} \frac{1 + e^{-i \pi m}}{2}
-
\sqrt{\frac{1}{M}} \sum_{m = 0}^{M - 1}
e^{-i \frac{2 \pi}{M} m j} x_{m} \frac{1 - e^{-i \pi m}}{2} 
=
\nonumber \\
=
\sqrt{\frac{1}{M}} \sum_{m = 0}^{M - 1}
e^{-i \frac{2 \pi}{M} m j} e^{-i \pi m } x_{m} 
=
\sqrt{\frac{1}{M}} \sum_{m = 0}^{M - 1}
e^{-i \frac{2 \pi}{M} m j} e^{-i \frac{2 \pi}{M} m \frac{M}{2} } x_{m} 
=
\nonumber \\
=
\sqrt{\frac{1}{M}} \sum_{m = 0}^{M - 1}
e^{-i \frac{2 \pi}{M} m \left(\frac{M}{2} + j\right)} x_{m}
\label{eqPart4QuantCompShorFourierStep3_2}
\end{eqnarray}

Combine (\ref{eqPart4QuantCompShorFourierStep3}), 
(\ref{eqPart4QuantCompShorFourierStep3_1}) and
(\ref{eqPart4QuantCompShorFourierStep3_2}) finally it can be got  
\begin{eqnarray}
\left|x\right> \rightarrow
\sum_{j = 0}^{\frac{M}{2} - 1} \sqrt{\frac{1}{M}} \sum_{m = 0}^{M - 1}
e^{-i \frac{2 \pi}{M} m j} x_{m} \left|j\right>_{inv} +
\nonumber \\
+
\sum_{j = 0}^{\frac{M}{2} - 1} \sqrt{\frac{1}{M}} \sum_{m = 0}^{M - 1}
e^{-i \frac{2 \pi}{M} m \left(\frac{M}{2} + j\right)} x_{m} 
\left|\frac{M}{2} + j\right>_{inv} =
\nonumber \\
= \sum_{j = 0}^{M - 1} \tilde{X}_j^{M} \left|j\right>_{inv}
\nonumber
\end{eqnarray}

Thus the final scheme (Fig. \ref{figQuantCompQuantFourier}d)
provides us the required Fourier transformation
(\ref{eqPart4QuantCompShorQuantFourierEnd1}).

\section{Conclusion}
As a result of the scheme analyze the proof of correctness was
got. The scheme analyze was divided into several steps. The first one
included Fourier transform for highest bits only. Unfortunately the
result state did not conform the requested one. Especially a phase is
not correct. The issue can be fixed with phase shift correction
elements that were added in the second step. As result the correct
Fourier transform was got for even components but not for odd
ones. The problem can be resolved by means of the Hadamard gate that was
added on the last step. It was shown that the final circuit performs
the requested Fourier transform.   

\begin{thebibliography}{}
   \bibitem{shor} Shor, P.: Algorithms for Quantum Computation:
     Discrete Logarithms and Factoring. FOCS, 124--134 (1994)
   \bibitem{kitaev} Kitaev A. Yu: Quantum measurements and the
     Abelian Stabilizer Problem, arXiv:quant-ph/9511026.
   \bibitem{lanyon} Lanyon, B. P.; Weinhold, T. J.; Langford,
     N. K.; Barbieri, M.; James, D. F. V.; Gilchrist, A.; White,
     A. G. (2007): Experimental Demonstration of a Compiled Version
     of Shor's Algorithm with Quantum Entanglement. Physical Review
     Letters 99 (25): 250505      
   \bibitem{ettinger} Mark Ettinger; Peter Høyer: A quantum
     observable for the graph isomorphism problem, arXiv:quant-ph/9901029 
   \bibitem{fft} Cooley, James W.; Tukey, John W. (1965): An
     algorithm for the machine calculation of complex Fourier
     series. Math. Comput. 19: 297-–301. doi:10.2307/2003354
\end{thebibliography}

\newpage
\begin{figure}
\ifx\JPicScale\undefined\def\JPicScale{1}\fi
\unitlength \JPicScale mm
\begin{picture}(127.57,94.5)(0,0)
\linethickness{0.3mm}
\put(4.61,79.5){\line(1,0){11.52}}
\linethickness{0.3mm}
\put(4.61,67){\line(1,0){11.52}}
\linethickness{0.3mm}
\put(4.61,62){\line(1,0){11.52}}
\linethickness{0.3mm}
\put(16.13,79.5){\line(1,0){11.53}}
\linethickness{0.3mm}
\put(27.66,79.5){\line(1,0){26.9}}
\linethickness{0.3mm}
\put(16.13,67){\line(1,0){23.06}}
\linethickness{0.3mm}
\put(39.19,67){\line(1,0){15.36}}
\linethickness{0.3mm}
\put(16.13,62){\line(1,0){28.82}}
\linethickness{0.3mm}
\put(44.95,62){\line(1,0){9.61}}
\linethickness{0.3mm}
\put(50.72,89.5){\line(1,0){3.84}}
\put(0.77,62){\makebox(0,0)[cc]{$\left| a_{n-1} \right>$}}

\put(0.77,67){\makebox(0,0)[cc]{$\left| a_{n -2} \right>$}}

\put(0.77,89.5){\makebox(0,0)[cc]{$\left| a_0 \right>$}}
\put(0.77,97.0){\makebox(0,0)[cc]{(a)}}

\put(58.4,62){\makebox(0,0)[cc]{$\left| b_{0} \right>$}}

\put(58.4,67){\makebox(0,0)[cc]{$\left| b_{1} \right>$}}

\put(59.4,89.5){\makebox(0,0)[cc]{$\left| b_{n-1} \right>$}}

\linethickness{0.3mm}
\put(6.53,94.5){\line(1,0){46.11}}
\put(6.53,57){\line(0,1){37.5}}
\put(52.64,57){\line(0,1){37.5}}
\put(6.53,57){\line(1,0){46.11}}
\put(10.0,92){\makebox(0,0)[cc]{$\hat{F}^{M}$}}

\linethickness{0.3mm}

\linethickness{0.3mm}

\linethickness{0.3mm}

\linethickness{0.3mm}

\linethickness{0.3mm}

\linethickness{0.3mm}

\linethickness{0.3mm}

\linethickness{0.3mm}

\linethickness{0.3mm}

\linethickness{0.3mm}

\linethickness{0.3mm}

\linethickness{0.3mm}

\linethickness{0.3mm}

\linethickness{0.3mm}

\linethickness{0.3mm}

\linethickness{0.3mm}

\linethickness{0.3mm}

\linethickness{0.3mm}
\put(4.61,89.5){\line(1,0){15.37}}
\linethickness{0.3mm}
\put(31.5,89.5){\line(1,0){19.21}}
\linethickness{0.3mm}

\linethickness{0.3mm}

\linethickness{0.3mm}

\linethickness{0.3mm}

\linethickness{0.3mm}

\linethickness{0.3mm}

\linethickness{0.3mm}

\put(0.77,79.5){\makebox(0,0)[cc]{$\left| a_k \right>$}}

\linethickness{0.3mm}

\linethickness{0.3mm}

\linethickness{0.3mm}

\put(61.0,79.5){\makebox(0,0)[cc]{$\left| b_{n - 1 - k} \right>$}}

\linethickness{0.3mm}
\put(19.98,89.5){\line(1,0){11.52}}
\linethickness{0.3mm}

\linethickness{0.3mm}

\linethickness{0.3mm}

\linethickness{0.3mm}
\put(104.52,24.5){\line(1,0){3.84}}
\put(104.52,19.5){\line(0,1){5}}
\put(108.36,19.5){\line(0,1){5}}
\put(104.52,19.5){\line(1,0){3.84}}
\linethickness{0.3mm}
\put(110.27,19.5){\line(1,0){3.84}}
\put(110.27,14.5){\line(0,1){5}}
\put(114.11,14.5){\line(0,1){5}}
\put(110.27,14.5){\line(1,0){3.84}}
\linethickness{0.3mm}
\put(73.77,34.5){\line(1,0){3.84}}
\linethickness{0.3mm}
\put(73.77,22){\line(1,0){3.84}}
\linethickness{0.3mm}
\put(73.77,17){\line(1,0){3.84}}
\linethickness{0.3mm}
\put(85.3,34.5){\line(1,0){7.68}}
\linethickness{0.3mm}
\put(96.82,34.5){\line(1,0){26.9}}
\linethickness{0.3mm}
\put(85.3,22){\line(1,0){19.21}}
\linethickness{0.3mm}
\put(108.36,22){\line(1,0){15.36}}
\linethickness{0.3mm}
\put(85.3,17){\line(1,0){24.97}}
\linethickness{0.3mm}
\put(114.11,17){\line(1,0){9.61}}
\linethickness{0.3mm}
\put(94.91,37){\line(0,1){7.5}}
\linethickness{0.3mm}
\put(106.43,24.5){\line(0,1){20}}
\linethickness{0.3mm}
\put(112.2,19.5){\line(0,1){25}}
\linethickness{0.3mm}
\put(95.35,44.27){\line(0,1){0.48}}
\multiput(95.09,43.93)(0.13,0.17){2}{\line(0,1){0.17}}
\put(94.72,43.93){\line(1,0){0.37}}
\multiput(94.46,44.27)(0.13,-0.17){2}{\line(0,-1){0.17}}
\put(94.46,44.27){\line(0,1){0.48}}
\multiput(94.46,44.74)(0.13,0.17){2}{\line(0,1){0.17}}
\put(94.72,45.08){\line(1,0){0.37}}
\multiput(95.09,45.08)(0.13,-0.17){2}{\line(0,-1){0.17}}

\linethickness{0.3mm}
\put(106.87,44.27){\line(0,1){0.48}}
\multiput(106.61,43.93)(0.13,0.17){2}{\line(0,1){0.17}}
\put(106.25,43.93){\line(1,0){0.37}}
\multiput(105.99,44.27)(0.13,-0.17){2}{\line(0,-1){0.17}}
\put(105.99,44.27){\line(0,1){0.48}}
\multiput(105.99,44.74)(0.13,0.17){2}{\line(0,1){0.17}}
\put(106.25,45.08){\line(1,0){0.37}}
\multiput(106.61,45.08)(0.13,-0.17){2}{\line(0,-1){0.17}}

\linethickness{0.3mm}
\put(112.64,44.27){\line(0,1){0.48}}
\multiput(112.38,43.93)(0.13,0.17){2}{\line(0,1){0.17}}
\put(112.01,43.93){\line(1,0){0.37}}
\multiput(111.75,44.27)(0.13,-0.17){2}{\line(0,-1){0.17}}
\put(111.75,44.27){\line(0,1){0.48}}
\multiput(111.75,44.74)(0.13,0.17){2}{\line(0,1){0.17}}
\put(112.01,45.08){\line(1,0){0.37}}
\multiput(112.38,45.08)(0.13,-0.17){2}{\line(0,-1){0.17}}

\linethickness{0.3mm}
\put(116.04,47){\line(1,0){3.84}}
\put(116.04,42){\line(0,1){5}}
\put(119.88,42){\line(0,1){5}}
\put(116.04,42){\line(1,0){3.84}}
\linethickness{0.3mm}
\put(119.88,44.5){\line(1,0){3.84}}
\put(117.96,44.5){\makebox(0,0)[cc]{$\hat{H}$}}

\put(69.93,17){\makebox(0,0)[cc]{$\left| a_{n-1} \right>$}}

\put(69.93,22){\makebox(0,0)[cc]{$\left| a_{n -2} \right>$}}

\put(69.93,44.5){\makebox(0,0)[cc]{$\left| a_0 \right>$}}
\put(69.93,52.5){\makebox(0,0)[cc]{(d)}}

\put(127.57,17){\makebox(0,0)[cc]{$\left| b_{0} \right>$}}

\put(127.57,22){\makebox(0,0)[cc]{$\left| b_{1} \right>$}}

\put(128.57,44.5){\makebox(0,0)[cc]{$\left| b_{n-1} \right>$}}

\put(81.0,37){\makebox(0,0)[cc]{$\hat{F}^{\frac{M}{2}}$}}

\linethickness{0.3mm}
\put(75.7,49.5){\line(1,0){46.11}}
\put(75.7,12){\line(0,1){37.5}}
\put(121.8,12){\line(0,1){37.5}}
\put(75.7,12){\line(1,0){46.11}}
\put(79.0,47){\makebox(0,0)[cc]{$\hat{F}^{M}$}}

\linethickness{0.3mm}

\linethickness{0.3mm}

\linethickness{0.3mm}

\linethickness{0.3mm}

\linethickness{0.3mm}

\linethickness{0.3mm}

\linethickness{0.3mm}

\linethickness{0.3mm}

\linethickness{0.3mm}

\linethickness{0.3mm}

\linethickness{0.3mm}

\linethickness{0.3mm}

\linethickness{0.3mm}

\linethickness{0.3mm}

\linethickness{0.3mm}

\linethickness{0.3mm}

\linethickness{0.3mm}

\linethickness{0.3mm}
\put(73.77,44.5){\line(1,0){15.37}}
\linethickness{0.3mm}
\put(100.67,44.5){\line(1,0){11.53}}
\linethickness{0.3mm}
\put(112.2,44.5){\line(1,0){3.84}}
\linethickness{0.3mm}

\linethickness{0.3mm}

\linethickness{0.3mm}

\linethickness{0.3mm}

\put(112.2,17){\makebox(0,0)[cc]{$\hat{R}_{0}$}}

\put(106.43,22){\makebox(0,0)[cc]{$\hat{R}_{1}$}}

\linethickness{0.3mm}
\put(77.61,40.75){\line(1,0){7.69}}
\put(77.61,13.25){\line(0,1){27.5}}
\put(85.3,13.25){\line(0,1){27.5}}
\put(77.61,13.25){\line(1,0){7.69}}
\linethickness{0.3mm}

\linethickness{0.3mm}

\linethickness{0.3mm}

\linethickness{0.3mm}

\put(69.93,34.5){\makebox(0,0)[cc]{$\left| a_{k} \right>$}}

\linethickness{0.3mm}

\linethickness{0.3mm}

\linethickness{0.3mm}

\put(130.0,34.5){\makebox(0,0)[cc]{$\left| b_{n - 1 - k} \right>$}}

\linethickness{0.3mm}
\put(92.98,37){\line(1,0){3.84}}
\put(92.98,32){\line(0,1){5}}
\put(96.82,32){\line(0,1){5}}
\put(92.98,32){\line(1,0){3.84}}
\linethickness{0.3mm}

%\put(94.91,34.5){\makebox(0,0)[cc]{$\hat{R}_k$}}

\linethickness{0.3mm}
\put(92.98,44.5){\line(1,0){3.84}}
\linethickness{0.3mm}
\put(73.77,79.5){\line(1,0){3.84}}
\linethickness{0.3mm}
\put(73.77,67){\line(1,0){3.84}}
\linethickness{0.3mm}
\put(73.77,62){\line(1,0){3.84}}
\linethickness{0.3mm}
\put(85.3,79.5){\line(1,0){11.52}}
\linethickness{0.3mm}
\put(96.82,79.5){\line(1,0){26.9}}
\linethickness{0.3mm}
\put(85.3,67){\line(1,0){23.05}}
\linethickness{0.3mm}
\put(108.36,67){\line(1,0){15.36}}
\linethickness{0.3mm}
\put(85.3,62){\line(1,0){28.81}}
\linethickness{0.3mm}
\put(114.11,62){\line(1,0){9.61}}
\linethickness{0.3mm}
\put(119.88,89.5){\line(1,0){3.84}}
\put(69.93,62){\makebox(0,0)[cc]{$\left| a_{n-1} \right>$}}

\put(69.93,67){\makebox(0,0)[cc]{$\left| a_{n -2} \right>$}}

\put(69.93,89.5){\makebox(0,0)[cc]{$\left| a_0 \right>$}}
\put(69.93,97.0){\makebox(0,0)[cc]{(b)}}

\put(127.57,62){\makebox(0,0)[cc]{$\left| b_{0} \right>$}}

\put(127.57,67){\makebox(0,0)[cc]{$\left| b_{1} \right>$}}

\put(128.57,89.5){\makebox(0,0)[cc]{$\left| b_{n-1} \right>$}}

\put(81.0,82){\makebox(0,0)[cc]{$\hat{F}^{\frac{M}{2}}$}}

\linethickness{0.3mm}
\put(75.7,94.5){\line(1,0){46.11}}
\put(75.7,57){\line(0,1){37.5}}
\put(121.8,57){\line(0,1){37.5}}
\put(75.7,57){\line(1,0){46.11}}
\put(79.0,92){\makebox(0,0)[cc]{$\hat{F}^{M}$}}

\linethickness{0.3mm}

\linethickness{0.3mm}

\linethickness{0.3mm}

\linethickness{0.3mm}

\linethickness{0.3mm}

\linethickness{0.3mm}

\linethickness{0.3mm}

\linethickness{0.3mm}

\linethickness{0.3mm}

\linethickness{0.3mm}

\linethickness{0.3mm}

\linethickness{0.3mm}

\linethickness{0.3mm}

\linethickness{0.3mm}

\linethickness{0.3mm}

\linethickness{0.3mm}

\linethickness{0.3mm}

\linethickness{0.3mm}
\put(73.77,89.5){\line(1,0){15.37}}
\linethickness{0.3mm}
\put(100.67,89.5){\line(1,0){19.21}}
\linethickness{0.3mm}

\linethickness{0.3mm}

\linethickness{0.3mm}

\linethickness{0.3mm}

\linethickness{0.3mm}
\put(77.61,85.75){\line(1,0){7.69}}
\put(77.61,58.25){\line(0,1){27.5}}
\put(85.3,58.25){\line(0,1){27.5}}
\put(77.61,58.25){\line(1,0){7.69}}
\linethickness{0.3mm}

\linethickness{0.3mm}

\linethickness{0.3mm}

\linethickness{0.3mm}

\put(69.93,79.5){\makebox(0,0)[cc]{$\left| a_k \right>$}}

\linethickness{0.3mm}

\linethickness{0.3mm}

\linethickness{0.3mm}

\put(130.0,79.5){\makebox(0,0)[cc]{$\left| b_{n - 1 - k} \right>$}}

\linethickness{0.3mm}

\linethickness{0.3mm}
\put(92.98,89.5){\line(1,0){3.84}}
\linethickness{0.3mm}
\put(35.34,24.5){\line(1,0){3.85}}
\put(35.34,19.5){\line(0,1){5}}
\put(39.19,19.5){\line(0,1){5}}
\put(35.34,19.5){\line(1,0){3.85}}
\linethickness{0.3mm}
\put(41.11,19.5){\line(1,0){3.84}}
\put(41.11,14.5){\line(0,1){5}}
\put(44.95,14.5){\line(0,1){5}}
\put(41.11,14.5){\line(1,0){3.84}}
\linethickness{0.3mm}
\put(4.61,34.5){\line(1,0){3.84}}
\linethickness{0.3mm}
\put(4.61,22){\line(1,0){3.84}}
\linethickness{0.3mm}
\put(4.61,17){\line(1,0){3.84}}
\linethickness{0.3mm}
\put(16.13,34.5){\line(1,0){7.69}}
\linethickness{0.3mm}
\put(27.66,34.5){\line(1,0){26.9}}
\linethickness{0.3mm}
\put(16.13,22){\line(1,0){19.21}}
\linethickness{0.3mm}
\put(39.19,22){\line(1,0){15.36}}
\linethickness{0.3mm}
\put(16.13,17){\line(1,0){24.98}}
\linethickness{0.3mm}
\put(44.95,17){\line(1,0){9.61}}
\linethickness{0.3mm}
\put(25.74,37){\line(0,1){7.5}}
\linethickness{0.3mm}
\put(37.27,24.5){\line(0,1){20}}
\linethickness{0.3mm}
\put(43.03,19.5){\line(0,1){25}}
\linethickness{0.3mm}
\put(26.19,44.27){\line(0,1){0.48}}
\multiput(25.93,43.93)(0.13,0.17){2}{\line(0,1){0.17}}
\put(25.56,43.93){\line(1,0){0.37}}
\multiput(25.29,44.27)(0.13,-0.17){2}{\line(0,-1){0.17}}
\put(25.29,44.27){\line(0,1){0.48}}
\multiput(25.29,44.74)(0.13,0.17){2}{\line(0,1){0.17}}
\put(25.56,45.08){\line(1,0){0.37}}
\multiput(25.93,45.08)(0.13,-0.17){2}{\line(0,-1){0.17}}

\linethickness{0.3mm}
\put(37.71,44.27){\line(0,1){0.48}}
\multiput(37.45,43.93)(0.13,0.17){2}{\line(0,1){0.17}}
\put(37.08,43.93){\line(1,0){0.37}}
\multiput(36.83,44.27)(0.13,-0.17){2}{\line(0,-1){0.17}}
\put(36.83,44.27){\line(0,1){0.48}}
\multiput(36.83,44.74)(0.13,0.17){2}{\line(0,1){0.17}}
\put(37.08,45.08){\line(1,0){0.37}}
\multiput(37.45,45.08)(0.13,-0.17){2}{\line(0,-1){0.17}}

\linethickness{0.3mm}
\put(43.48,44.27){\line(0,1){0.48}}
\multiput(43.22,43.93)(0.13,0.17){2}{\line(0,1){0.17}}
\put(42.85,43.93){\line(1,0){0.37}}
\multiput(42.59,44.27)(0.13,-0.17){2}{\line(0,-1){0.17}}
\put(42.59,44.27){\line(0,1){0.48}}
\multiput(42.59,44.74)(0.13,0.17){2}{\line(0,1){0.17}}
\put(42.85,45.08){\line(1,0){0.37}}
\multiput(43.22,45.08)(0.13,-0.17){2}{\line(0,-1){0.17}}

\linethickness{0.3mm}
\put(50.72,44.5){\line(1,0){3.84}}
\put(0.77,17){\makebox(0,0)[cc]{$\left| a_{n-1} \right>$}}

\put(0.77,22){\makebox(0,0)[cc]{$\left| a_{n -2} \right>$}}

\put(0.77,44.5){\makebox(0,0)[cc]{$\left| a_0 \right>$}}
\put(0.77,52.5){\makebox(0,0)[cc]{(c)}}

\put(58.4,17){\makebox(0,0)[cc]{$\left| b_{0} \right>$}}

\put(58.4,22){\makebox(0,0)[cc]{$\left| b_{1} \right>$}}

\put(59.4,44.5){\makebox(0,0)[cc]{$\left| b_{n - 1} \right>$}}

\put(12.0,37){\makebox(0,0)[cc]{$\hat{F}^{\frac{M}{2}}$}}

\linethickness{0.3mm}
\put(6.53,49.5){\line(1,0){46.11}}
\put(6.53,12){\line(0,1){37.5}}
\put(52.64,12){\line(0,1){37.5}}
\put(6.53,12){\line(1,0){46.11}}
\put(10.0,47){\makebox(0,0)[cc]{$\hat{F}^{M}$}}

\linethickness{0.3mm}

\linethickness{0.3mm}

\linethickness{0.3mm}

\linethickness{0.3mm}

\linethickness{0.3mm}

\linethickness{0.3mm}

\linethickness{0.3mm}

\linethickness{0.3mm}

\linethickness{0.3mm}

\linethickness{0.3mm}

\linethickness{0.3mm}

\linethickness{0.3mm}

\linethickness{0.3mm}

\linethickness{0.3mm}

\linethickness{0.3mm}

\linethickness{0.3mm}

\linethickness{0.3mm}

\linethickness{0.3mm}
\put(4.61,44.5){\line(1,0){15.37}}
\linethickness{0.3mm}
\put(31.5,44.5){\line(1,0){11.53}}
\linethickness{0.3mm}
\put(43.03,44.5){\line(1,0){7.68}}
\linethickness{0.3mm}

\linethickness{0.3mm}

\linethickness{0.3mm}

\linethickness{0.3mm}

\put(43.03,17){\makebox(0,0)[cc]{$\hat{R}_{0}$}}

\put(37.27,22){\makebox(0,0)[cc]{$\hat{R}_{1}$}}

\linethickness{0.3mm}
\put(8.45,40.75){\line(1,0){7.68}}
\put(8.45,13.25){\line(0,1){27.5}}
\put(16.13,13.25){\line(0,1){27.5}}
\put(8.45,13.25){\line(1,0){7.68}}
\linethickness{0.3mm}

\linethickness{0.3mm}

\linethickness{0.3mm}

\linethickness{0.3mm}

\put(0.77,34.5){\makebox(0,0)[cc]{$\left| a_{k} \right>$}}

\linethickness{0.3mm}

\linethickness{0.3mm}

\linethickness{0.3mm}

\put(61.0,34.5){\makebox(0,0)[cc]{$\left| b_{n-1-k} \right>$}}

\linethickness{0.3mm}
\put(23.82,37){\line(1,0){3.84}}
\put(23.82,32){\line(0,1){5}}
\put(27.66,32){\line(0,1){5}}
\put(23.82,32){\line(1,0){3.84}}
\linethickness{0.3mm}

%\put(25.74,34.5){\makebox(0,0)[cc]{$\hat{R}_{n-k-1}$}}

\linethickness{0.3mm}
\put(23.82,44.5){\line(1,0){3.84}}
\end{picture}
\caption{
  The quantum Fourier transform circuit based on the fast
  Fourier transform algorithm.
  Initial data on the input (a).
  Step 1 (b): 
  $\left|x\right> \rightarrow
  \hat{F}^{\frac{M}{2}} \sum_{m = 0}^{\frac{M}{2} - 1}x_{2m} \left|2m\right> +
  \hat{F}^{\frac{M}{2}} \sum_{m = 0}^{\frac{M}{2} - 1}x_{2m + 1}
  \left|2m+1\right>$.
  Step 2 (c): 
  $\left|x\right> \rightarrow
  \hat{F}^{\frac{M}{2}} \sum_{m = 0}^{\frac{M}{2} - 1}x_{2m} \left|2m\right> +
  \hat{R}\hat{F}^{\frac{M}{2}} \sum_{m = 0}^{\frac{M}{2} - 1} x_{2m + 1}$.
  Final circuit (d).
}
\label{figQuantCompQuantFourier}
\end{figure}

\end{document}

\section{DISCUSSION.}

\noindent The derived equations set is suitable for the numerical
simulation. We've considered case of resonator with two allocated
modes in contrast to early article \cite{ourSemi}. These modes differ
from each other by means of coefficient $B^{(i)}$ (depends on a mode
distributions) and initial seeds $M_0^{(i)}$. First mode is allocated
by seed 
\[
M_0^{(1)} > M_0^{(2)},
\]
second is allocated by $B$ coefficient:
\[
B^{(2)} > B^{(1)}.
\]

\input ./fig_num.tex

The result of  numerical calculation  is
presented on Fig. \ref{figNum}.  We can see ``landing''  that take
priority of steady-state 
condition. We've sought to find explanation. Dependencies of modes
(first and second) from time were calculated for interpretation (see
Fig. \ref{figModes}). We can see that ``landing'' corresponds to  steady-state
condition for first mode ($t < 80$). Second mode is weak at this case
($t < 80$). The second mode began to increase at $t \approx 80$. The
first mode began to drop at this moment. The first mode is depressed
by second mode at steady-state condition ($t > 100$).  

\input ./fig_num_modes.tex

\input ./fig_exp.tex

A typical result of the experimental measurements is shown on the
Fig. \ref{figExp}. We tried to reproduce the situation of numerical
calculations. We can see two ``landings'' on Fig. \ref{figExp}. Also we
have oscillation mode at steady-state condition (Fig. \ref{figExp})
that corresponds to numerical calculations Fig. \ref{figNum}.

Similar experiments where the seed was modified were made in
\cite{Zozulya}. We also can see ``landings'' in this case.
The length of  ``landings'' depended from initial
seed and increased with decrease of seed. We can explain this
fact in our model by gain of steady-state time for second mode.\\

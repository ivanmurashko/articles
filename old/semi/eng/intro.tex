\title{Mode competition in the semilinear generator geometry.} 

\author{I.V.Murashko, \fbox{Ph. N. Nikiforov}, V.Yu.Petrun'kin, I.A.Vodovatov.
\skiplinehalf St.Petersburg State Technical University,\\
Radiophysical faculty, dept. of Quantum Electronics, 195251,\\
Polytekhnicheskaya, 29, St.Petersburg, Russia}

\authorinfo{You may contact us via e-mail: ipv@quantel.stu.neva.ru}

\begin{document}
\maketitle

\begin{abstract}
{Self-diffraction in the semilinear generator geometry is
investigated using statistical approach for multimode model.
The system of equation that describes wave conjugation for multimode
model is obtained. Comparison of the numerical
modelling results and physical experiments in the geometry of
semilinear generator with  BSO crystals shows good qualitative
concordance.} 
\end{abstract}

\keywords{Semilinear generator, phase conjugation,
self-diffraction, mode competition.}

\section{INTRODUCTION.}

\noindent The process of the conjugated waves build-up is an
interesting observable phenomenon in the four wave mixing
experiments, which was extensively described
before\cite{Horowitz1991,Bel'dyugina1992,Dynamics}. This process can
be used for construction of optical neural networks\cite{Psaltis,ART}.
Semilinear
generator is a popular scheme that is used for this. There are many
works analyse this scheme long time\cite{bFirstSemi,Zozulya}. The main
limitation of these paper was oversimplified mathematical model used
for modelling conjugation process. For example fanning 
influence was ignored in majority papers, also oversimplified case of
plane light beams was considered. 

We had developed the math model\cite{ourDPCM} that considered the fanning
influence. Also we had expected that light beams that interacted in a
photorefractive crystal are light beams with complex angle
distribution. In next paper  \cite{ourSemi} we theorised this
model on semilinear generator geometry. 
The purpose of the present investigation was to exploration semilinear
generator geometry in multimode case.\\

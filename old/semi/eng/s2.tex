
\section{DERIVATION OF THE EQUATIONS.}

\noindent The geometry of the semilinear generator is
presented on the Fig. 1. In the given geometry, the pump wave from
one side (at the left) is supplied by a laser, and on the other
side (on the right) the pump is the wave reflected from the
resonator's mirror.

\begin{figure}
\centering
\input semi1_pic.tex
\label{semi} \caption{\footnotesize Geometry of the semilinear
generator: experimental setup. BS - Beam Splitter, BSO -
photorefractive crystal. $\vec{E_1}$ waves propagate from the left
to the right, $\vec{E_2}$ waves do the opposite way. The signal
wave is supposed to have complex structure after passing through
the transparent plate T.}
\end{figure}

The theory of the semilinear generator was developed
proceeding from the theory of the double phase conjugation mirror
\cite{ourDPCM} at our earlier paper \cite{ourSemi}. The system of
equation \cite{ourSemi} was deduced for singlemode oscillations.  The
main goal  of the present investigation was to research semilinear
generator with multimode oscillations. 

\input ./fig_td3_gen.tex

In addition to those assumptions which were accepted in
\cite{ourSemi}, we assume, that field $\vec{E_2}$ can be expanded over
the system of the resonator modes. Let's assume that the phase
distribution of a mode is $\vec{e_2}^{(i)}$, where $i$ is the index of
the mode. Also we'll assume that some modes are
allocated. Thus field $\vec{E_2}$ is  described by the following
equation (see Fig.
\ref{figTD3_gen}):
\[ 
\vec{E}_2=\sum_i \vec{E}_{2s}^{(i)} +\vec{E}_{2c} + \vec{E}_{2d} + \vec{E}_{2f}, 
\] 
\noindent where $\vec{E}_{2s}^{(i)}$ is a set of allocated modes
(signal wave), all
non-allocated modes considered in $\vec{E}_{2d}$ (with distribution
$\vec{e_d}$),  $\vec{E}_{2c}$ and
$\vec{E}_{2f}$ can are described by analogy with \cite{ourSemi}.

\input ./fig_td2_gen.tex

We write the field on the left as well as \cite{ourSemi}(see Fig.
\ref{figTD2_gen}):
\[
\vec{E}_1=\vec{E}_{1s}+\sum_i \vec{E}_{1c}^{(i)}+\vec{E}_{1f}, 
\]
\noindent where $\vec{E}_{1s}$ - is the signal wave having
structure of a wave falling on the crystal at the left (with
distribution $\vec {e_1}$); $\vec {E}_{1c}^{(i)}$ is the set of
conjugated waves (resonator's modes), $\vec {E}_{1f}$ has the meaning 
of the rest which is a wave scattered on
the on the random inhomogeneities of the refraction index.

The fields contained in $\vec {E}_1$, we consider in the following form:
\[
\vec{E}_{1s}=A_{1s}(z,t)\vec{e}_1(\vec{r})e^{-i(\vec{k}_1\vec{r})},
\]
\[
\vec{E}_{1c}^{(i)}=A^{(i)}_{1c}(z,t)\vec{e}_2^{(i)\ast}
(\vec{r})e^{i(\vec{k}_2\vec{r})}, 
\]
\[
\vec{E}_{1f}=A_{1f}(z,t)\vec{e}_{1f}(\vec{r},t)e^{i(\vec{k}_2\vec{r})}, 
\]
\noindent where $\vec{e}_1(\vec{r})$ - is the distribution of the
undistorted (signal) wave incident from the left side of the
crystal; $\vec{e}_2(\vec{r})$ is the distribution of the signal
wave from the right side; $\vec{e}_{1f}(\vec{r},t)$ is the
distribution of fanning.

The components of $\vec{E}_2$ can be written similarly:
\[
\vec{E}_{2s}^{(i)}=A^{(i)}_{2s}(z,t)\vec{e}_2^{(i)}(\vec{r})
e^{-i(\vec{k}_2\vec{r})},
\]
\[
\vec{E}_{2c}=A_{2c}(z,t)\vec{e}_1^\ast(\vec{r})e^{i(\vec{k}_1\vec{r})}, 
\]
\[
\vec{E}_{2d}=A_{2d}(z,t)\vec{e}_d(\vec{r})e^{-i(\vec{k}_2\vec{r})}, 
\]
\[
\vec{E}_{2f}=A_{2f}(z,t)\vec{e}_{2f}(\vec{r},t)e^{i(\vec{k}_1\vec{r})}.
\]

Thus we can get the following system of equations by analogy with
\cite{ourSemi}.
First is a set of material equations:
\begin{eqnarray}  
\left(\frac{\partial}{\partial{t}} 
+\frac{1}{\tau_{eff}}
\left(
1 + i \frac{E_D}{E_q}
\right)
\right) M^{(i)}
= 
i \Gamma B^{(i)} 
\frac{A_{1s}^{\ast}A_{1c}^{(i)}+A_{2s}^{(i)\ast}A_{2c}}{I_0}
+ \frac{M_0^{(i)}}{\tau_{eff}},
\nonumber \\
\left(\frac{\partial}{\partial{t}} 
+\frac{1}{\tau_{eff}}
\left(
1 + i \frac{E_D}{E_q}
\right) 
\right) 
{M_{1f}} = 
i\Gamma C_1 \frac{A_{1s}^{\ast}A_{1f}}{I_0}
+ \frac{M_{1f_0}}{\tau_{eff}},
\nonumber \\
\left(\frac{\partial}{\partial{t}} 
+\frac{1}{\tau_{eff}}
\left(
1 + i \frac{E_D}{E_q}
\right) 
\right) 
{M_{2f}^{(i)}} = 
i\Gamma C_2^{(i)} \frac{A_{2s}^{(i)\ast}A_{2f}}{I_0}
+ \frac{M_{2f_0}^{(i)}}{\tau_{eff}},
\nonumber \\
\left(\frac{\partial}{\partial{t}} 
+\frac{1}{\tau_{eff}}
\left(
1 + i \frac{E_D}{E_q}
\right) 
\right) 
{M_d} = 
i \Gamma C_d \frac{A_{2c} A_{2d}^{\ast}}{I_0}
+ \frac{M_{d_0}}{\tau_{eff}},
\nonumber \\
\left(\frac{\partial}{\partial{t}} 
+\frac{1}{\tau_{eff}}
\left(
1 + i \frac{E_D}{E_q}
\right) 
\right) 
{M_{fd}} = 
i \Gamma C_{fd} \frac{A_{2f} A_{2d}^{\ast}}{I_0} +
\frac{M_{fd_0}}{\tau_{eff}},
\label{eqGrating_gen} 
\end{eqnarray} 
\noindent where coefficients $B^{(i)}$, $C_1$ and $C_2^{(i)}$ (see
\cite{ourSemi}) meet the conditions: 
\[
B^{(i)} = \left.const\right|_t < 1,
C_1 = \left.const\right|_t < 1,
C_2^{(i)} = \left.const\right|_t < 1.
\]
These  coefficients
depend upon the spatial distributions of the light fields as well as \cite{ourSemi}. The
value of $B^{(i)}$ is determined by $\vec{e}_1$ and $\vec{e}_2^{(i)}$. The
values of $C_1$ and $C_2^{(i)}$ can be estimated taking into account
reasonable assumptions concerning the fanning distributions.

Second is a system that describes $\vec E_1$ field:
\begin{eqnarray}
\frac{\partial{A_{1s}}}{\partial{z}} = 
- i D \left\{ 
 A_{1f} M_{1f}^\ast + \sum_i A_{1c}^{(i)}M^{(i)\ast}\right\},
\nonumber \\
\frac{\partial{A_{1c}^{(i)}}}{\partial{z}} = - 
i D A_{1s} M^{(i)},
\nonumber \\
\frac{\partial{A_{1f}}}{\partial{z}} = 
- i D A_{1s} M_{1f}.
\label{eqAmpl_left_gen}
\end{eqnarray}

Third is a system that describes $\vec E_2$ field:
\begin{eqnarray}
\frac{\partial{A_{2c}}}{\partial{z}} = 
i D \left\{ 
 A_{2d} M_d + \sum_i A_{2s}^{(i)} M^{(i)} \right\},
\nonumber \\
\frac{\partial{A_{2s}^{(i)}}}{\partial{z}} = 
i D \left\{
A_{2c} M^{(i)\ast}
+
A_{2f} M_{2f}^{(i)\ast}
\right\}, 
\nonumber \\
\frac{\partial{A_{2d}}}{\partial{z}} = 
i D \left\{
A_{2c} M_d^\ast
+
A_{2f} M_{fd}^\ast
\right\},
\nonumber \\
\frac{\partial{A_{2f}}}{\partial{z}} = 
i D  \left\{ 
 A_{2d} M_{fd} + \sum_i A_{2s}^{(i)} M_{2f}^{(i)} \right\}.
\label{eqAmpl_right_gen}
\end{eqnarray}

The initial conditions of the equation set are zero ones. The
boundary conditions from left are the following:
\begin{eqnarray}
\left. A_{1s} \right|_{z=0} = A_{1s_0},
\nonumber \\
\left. A_{1c}^{(i)} \right|_{z=0} = 0,
\nonumber \\
\left. A_{1f} \right|_{z=0} = 0.
\label{eqBoundary_left_gen}
\end{eqnarray}
Boundary condition from right side are the following:
\begin{eqnarray}
\left. A_{2s}^{(i)} \right|_{z=d} = r \left. A_{1c}^{(i)} \right|_{z=d},
\nonumber \\ 
\left. A_{2c} \right|_{z=d} = 0,
\nonumber \\ 
\left. A_{2d} \right|_{z=d} = r \left. A_{1f} \right|_{z=d},
\nonumber \\ 
\left. A_{2f} \right|_{z=d} = 0,
\label{eqBoundary_right_gen}
\end{eqnarray}
where $r$ take into account resonator losses.



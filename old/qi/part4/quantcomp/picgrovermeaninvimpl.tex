%%Created by jPicEdt 1.4.1_03: mixed JPIC-XML/LaTeX format
%%Mon Apr 01 14:33:06 MSD 2013
%%Begin JPIC-XML
%<?xml version="1.0" standalone="yes"?>
%<jpic x-min="0" x-max="120" y-min="0" y-max="40" auto-bounding="true">
%<multicurve fill-style= "none"
%	 points= "(5,35);(5,35);(20,35);(20,35)"
%	 />
%<multicurve fill-style= "none"
%	 points= "(5,15);(5,15);(20,15);(20,15)"
%	 />
%<multicurve fill-style= "none"
%	 points= "(5,5);(5,5);(50,5);(50,5)"
%	 />
%<parallelogram fill-style= "none"
%	 p3= "(70,0)"
%	 p2= "(70,40)"
%	 p1= "(50,40)"
%	 />
%<multicurve fill-style= "none"
%	 points= "(70,5);(70,5);(115,5);(115,5)"
%	 />
%<multicurve fill-style= "none"
%	 points= "(100,35);(100,35);(115,35);(115,35)"
%	 />
%<multicurve fill-style= "none"
%	 points= "(100,15);(100,15);(115,15);(115,15)"
%	 />
%<parallelogram fill-style= "none"
%	 p3= "(40,10)"
%	 p2= "(40,40)"
%	 p1= "(20,40)"
%	 />
%<parallelogram fill-style= "none"
%	 p3= "(100,10)"
%	 p2= "(100,40)"
%	 p1= "(80,40)"
%	 />
%<multicurve fill-style= "none"
%	 points= "(40,35);(40,35);(50,35);(50,35)"
%	 />
%<multicurve fill-style= "none"
%	 points= "(40,15);(40,15);(50,15);(50,15)"
%	 />
%<multicurve fill-style= "none"
%	 points= "(70,35);(70,35);(80,35);(80,35)"
%	 />
%<multicurve fill-style= "none"
%	 points= "(70,15);(70,15);(80,15);(80,15)"
%	 />
%<text fill-style= "none"
%	 text-vert-align= "center-v"
%	 anchor-point= "(60,35)"
%	 text-frame= "noframe"
%	 text-hor-align= "center-h"
%	 >
%$\hat{U}_{x \ne 0}$
%</text>
%<text fill-style= "none"
%	 text-vert-align= "center-v"
%	 anchor-point= "(30,35)"
%	 text-frame= "noframe"
%	 text-hor-align= "center-h"
%	 >
%$\hat{H}^{\otimes n}$
%</text>
%<text fill-style= "none"
%	 text-vert-align= "center-v"
%	 anchor-point= "(90,35)"
%	 text-frame= "noframe"
%	 text-hor-align= "center-h"
%	 >
%$\hat{H}^{\otimes n}$
%</text>
%<text fill-style= "none"
%	 text-vert-align= "center-v"
%	 anchor-point= "(0,25)"
%	 text-frame= "noframe"
%	 text-hor-align= "center-h"
%	 >
%$\left|\psi\right&gt;$
%</text>
%<text fill-style= "none"
%	 text-vert-align= "center-v"
%	 anchor-point= "(0,5)"
%	 text-frame= "noframe"
%	 text-hor-align= "center-h"
%	 >
%$\left|-\right&gt;$
%</text>
%<text fill-style= "none"
%	 text-vert-align= "center-v"
%	 anchor-point= "(120,5)"
%	 text-frame= "noframe"
%	 text-hor-align= "center-h"
%	 >
%$\left|-\right&gt;$
%</text>
%<text fill-style= "none"
%	 text-vert-align= "center-v"
%	 anchor-point= "(120,25)"
%	 text-frame= "noframe"
%	 text-hor-align= "center-h"
%	 >
%$\left|\psi^{\ast}\right&gt;$
%</text>
%</jpic>
%%End JPIC-XML
%LaTeX-picture environment using emulated lines and arcs
%You can rescale the whole picture (to 80% for instance) by using the command \def\JPicScale{0.8}
\ifx\JPicScale\undefined\def\JPicScale{1}\fi
\unitlength \JPicScale mm
\begin{picture}(120,40)(0,0)
\linethickness{0.3mm}
\put(5,35){\line(1,0){15}}
\linethickness{0.3mm}
\put(5,15){\line(1,0){15}}
\linethickness{0.3mm}
\put(5,5){\line(1,0){45}}
\linethickness{0.3mm}
\put(50,40){\line(1,0){20}}
\put(50,0){\line(0,1){40}}
\put(70,0){\line(0,1){40}}
\put(50,0){\line(1,0){20}}
\linethickness{0.3mm}
\put(70,5){\line(1,0){45}}
\linethickness{0.3mm}
\put(100,35){\line(1,0){15}}
\linethickness{0.3mm}
\put(100,15){\line(1,0){15}}
\linethickness{0.3mm}
\put(20,40){\line(1,0){20}}
\put(20,10){\line(0,1){30}}
\put(40,10){\line(0,1){30}}
\put(20,10){\line(1,0){20}}
\linethickness{0.3mm}
\put(80,40){\line(1,0){20}}
\put(80,10){\line(0,1){30}}
\put(100,10){\line(0,1){30}}
\put(80,10){\line(1,0){20}}
\linethickness{0.3mm}
\put(40,35){\line(1,0){10}}
\linethickness{0.3mm}
\put(40,15){\line(1,0){10}}
\linethickness{0.3mm}
\put(70,35){\line(1,0){10}}
\linethickness{0.3mm}
\put(70,15){\line(1,0){10}}
\put(60,35){\makebox(0,0)[cc]{$\hat{U}_{x \ne 0}$}}

\put(30,35){\makebox(0,0)[cc]{$\hat{H}^{\otimes n}$}}

\put(90,35){\makebox(0,0)[cc]{$\hat{H}^{\otimes n}$}}

\put(0,25){\makebox(0,0)[cc]{$\left|\psi\right>$}}

\put(0,5){\makebox(0,0)[cc]{$\left|-\right>$}}

\put(120,5){\makebox(0,0)[cc]{$\left|-\right>$}}

\put(120,25){\makebox(0,0)[cc]{$\left|\psi^{\ast}\right>$}}

\end{picture}

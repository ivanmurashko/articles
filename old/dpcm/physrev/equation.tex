\section{Derivation of equations}
\subsection{Impurity centers in photorefractive crystals}
Impurity centers have a big effect on the dynamic gratings formation in
photorefractive crystals. We will consider the following model of
photorefractive effect. Two coherent light beams are superimposed in a
photorefractive crystal and as result we have an interference pattern,
i.e. spatial space distribution with are of high and low intensity of
the light inside photorefractive crystal. Photoelectrons are excited
from  impurity centers $N_D$ 
in the areas of hight intensity. The photoelectrons migrate through
crystal by means of diffusion and trapped at donator cites $N_D^{+}$ in the
crystal. The process produces an inhomogeneous distribution of electric
charges and as result the electric space change field. The
electro-optic effect invokes the space distribution of refractive
index. 

There are several parameters that are used for impurity centers
description. The most important are the photoionization
cross-section $s$ and the recombination constant $\gamma_R$. The
photoionization 
cross-section $s$ is a characteristic of photoelectrons exciting
process. The recombination constant $\gamma_R$ is a characteristic of
recombination process.

The additional parameters that are used for impurity centers
description are concentration of impurity centers $N_D$ and acceptors
$N_A$. The necessity of $N_A$ introducing is the following. There are
ionized impurity centers at the dark conditions thus
\(\left.N_D^{+}\right|_{I=0} \ne 
0\). But the crystal is neutral itself, thus there have to be
additional impurity centers $N_A$ that compensate $N_D^{+}$ at the
dark conditions \(N_A = \left.N_D^{+}\right|_{I=0}\).

Fig.~\ref{fig:td_phr} shows two types
of crystal that we consider. The first one is a photorefractive
crystal with one impurity center (left figure) and the second one with
2 impurity center (right figure).

\input ./fig_td_phr.tex

The electric space change field $E_{sc}$ can be described by the following
equation in the case of one impirity center:
\begin{eqnarray}
\label{eqEnd_1_phr}
\tau_{eff} \frac {\partial E_{sc}} {\partial t} =
- i E_D \left(m + m_0\right) 
- \nonumber \\ 
- E_{sc} 
\left( 1 - i \frac{E_D}{E_q}\right),
\end{eqnarray}
where $E_D$ is the diffusion field, $E_q$ is the maximum
saturation field.  The $m_0$ describes is initial refractive index grating that
always presents in a photorefractive crystal.

The refractive index can be gotten from space charge field $E_{sc}$ 
with Pockels equation
\begin{equation}
\delta \varepsilon = \alpha E_{sc},
\label{eqElectroOpt_phr}
\end{equation}
where $\alpha$ is the electro-optical coefficient.

There are several typical times that describe the system. The speed of
photoelectrons excitation is described by $tau_I$:
\[
\tau_I = \frac{1}{s I_0 + \gamma_R n_0},
\]
where $I_0$ the light intensity, $n_0$ is the photoelectrons
density at the conduction band:
\[
n_0 = \frac{s I (N_D - N_A)}{\gamma_R N_A}.
\]
The photoelectrons recombination process is described by $tau_R$:
\[
\tau_R = \frac{1}{\gamma_R N_A}.
\]
In addition to $tau_I$ and $tau_R$ we introduce the Maxwell relaxation
time $\tau_m$: 
\[
\tau_m = \frac{\varepsilon}{4 \pi \mu n_0}
\]
and a value $tau_D$ that describes the diffusion process 
\[
\tau_D = \frac{1}{K^2 D},
\]
where $D$ is the coefficient of diffusion and $K$ is the propagation
vector of refractive index grating. 

At (\ref{eqEnd_1_phr}) we introduced a time that describes the system
integrally:
\[
\tau_{eff} = \tau_m \frac{\tau_R}{\tau},
\]
where for $\tau$ we have:
\[
\frac{1}{\tau} = \frac{1}{\tau_I} + \frac{1}{\tau_R} + \frac{1}{\tau_D}.
\]

For a crystal with 2 impurity centers the space charge field $E_{sc}$
is a sum of two fields that describe influence of different impurity
centers: 
\[
E_{sc} =  E_{sc}^{(1)} + E_{sc}^{(2)},
\]
For main class of impurity centers $E_{sc}^{(1)}$ we have the following
equation:  
\begin{eqnarray}
\label{eqEsc_1_End_2_phr}
\tau_{eff} 
\frac {\partial E_{sc}^{(1)}} {\partial t} =
- i E_D \left(m + m_0^{(1)}\right) 
- \nonumber \\ 
- E_{sc}^{(1)} 
\left( 1 - i \frac{E_D}{E_q^{(1)}} \right)
 -  E_{sc}^{(2)} 
\left( 1+ i \frac{E_D}{E_m^{(1)}} 
\right),
\nonumber \\ 
\left. {E_{sc}}^{(1)} \right|_{t=0} = 0.
\end{eqnarray}
For the second one $E_{sc}^{(2)}$ we have the following:
\begin{eqnarray}
\label{eqEsc_2_End_2_phr}
\frac {\partial E_{sc}^{(2)}} {\partial t} =
- \frac{i \left( m+ m_0^{(2)} \right) E_m^{(2)}}{\tau_m^{(2)}}
- \frac{E_{sc}^{(2)}}{\tau_I^{(2)}},
\nonumber \\ 
\left. {E_{sc}}^{(2)} \right|_{t=0} = 0.
\end{eqnarray}

We introduced a new parameter $E_m^{(1,2)}$ at (\ref{eqEsc_1_End_2_phr}) and
(\ref{eqEsc_2_End_2_phr}) that is defined by
\[
E_m^{(1,2)} = \frac{\gamma_R^{(1,2)} N_{A_0}^{(1,2)}}{K \mu}.
\] 

\subsection{Problem statement}

\input ./fig_td1_dpcm.tex

We will consider a double phase conjugate mirror model. At the model
two incoherent light beams drop on a photorefractive 
crystal from different sides as it is shown at
Fig.~\ref{fig:td1_dpcm}. As well as the light beams are incoherent
they can be considered independently.

\input ./fig_td2_dpcm.tex

The light field in photorefractive crystal is represented in the
following form (see Fig.~\ref{fig:td2_dpcm}):
\[ 
\begin{array}{l} 
\vec{E}_1=\vec{E}_{1s}+\vec{E}_{1c}+\vec{E}_{1f},\\ 
\vec{E}_2=\vec{E}_{2s}+\vec{E}_{2c}+\vec{E}_{2f}, 
\end{array} 
\] 
where $\vec{E}_{1s}$ is the signal wave that has a structure of the
light beam drops on the crystal from the left side; $\vec{E}_{1c}$ is the
conjugated wave that has a structure of the
light beam drops on the crystal from the right side; $\vec{E}_{1f}$ is
the rest that can be considered as a wave is scattered on the
crystal's inhomogeneities. The components of 
$\vec{E}_2$ that propagates from right side of the crystal have the
same meaning.

Assume
\[
\vec{E}_{1s}=A_{1s}(z,t)\vec{e}_1(\vec{r})e^{-i(\vec{k}_1\vec{r})},
\]
\[
\vec{E}_{1c}=A_{1c}(z,t)\vec{e}_2^\ast(\vec{r})e^{i(\vec{k}_2\vec{r})}, 
\]
\[
\vec{E}_{1f}=A_{1f}(z,t)\vec{e}_{1f}(\vec{r},t)e^{i(\vec{k}_2\vec{r})}, 
\]
where $\vec{e}_1(\vec{r})$ is the distribution for a wave drops on the
crystal from the left side; $\vec{e}_2(\vec{r})$ is the distribution for a
wave drops on the crystal from the right side; $\vec{e}_{1f}(\vec{r})$
is the distribution for stochastically scattered light.

It should be mentioned that gain for plane waves are different in
different directions. Thus the distribution function
$\vec{e}_{1f}(\vec{r})$ has a maximum in maximum gain direction.
We assume that the initial light beams directions are chosen as the
conjugated beams propagate in maximum gain direction. Therefore we
choose one propagation vector $\vec{k}_2$ for both $\vec{E}_{1c}$ and
$\vec{E}_{1f}$. 

Equations for $\vec{E}_2$ are same:
\[
\vec{E}_{2s}=A_{2s}(z,t)\vec{e}_2(\vec{r})e^{-i(\vec{k}_2\vec{r})},
\]
\[
\vec{E}_{2c}=A_{2c}(z,t)\vec{e}_1^\ast(\vec{r})e^{i(\vec{k}_1\vec{r})}, 
\]
\[
\vec{E}_{2f}=A_{2f}(z,t)\vec{e}_{2f}(\vec{r},t)e^{i(\vec{k}_1\vec{r})}, 
\]

The distribution functions $\vec{e}_1(\vec{r})$, $\vec{e}_2(\vec{r})$ and 
$\vec{e}_{1f,2f}(\vec{r})$ should be normalized:
\begin{eqnarray}
\int \limits_{S} \left( \vec{e}_1 \vec{e}_1^\ast \right) ds=1, 
\int \limits_{S} \left( \vec{e}_2 \vec{e}_2^\ast \right) ds=1,
\nonumber \\
\left\langle \int \limits_{S} \left(\vec{e}_{1f} 
\vec{e}_{1f}^{\ast} \right) ds \right\rangle = 1,
\left\langle \int \limits_{S} \left(\vec{e}_{2f} 
\vec{e}_{2f}^{\ast} \right) ds \right\rangle = 1,
\label{eqNorm_dpcm} 
\end{eqnarray}
where $\left\langle\right\rangle$ means ensemble averaging. In addition 
\begin{eqnarray} 
\left\langle e_{1f} \right\rangle = 0,
\left\langle e_{2f} \right\rangle = 0.
\label{eqZeroF_dpcm} 
\end{eqnarray}
The light beam from the left side of the crystal has the following form
\begin{equation} 
\vec{E}_1=A_{1s}\vec{e}_1 e^{-i(\vec{k}_1 \vec{r})} + 
A_{1c}\vec{e}_2^\ast e^{i(\vec{k}_2 \vec{r})} + 
A_{1f}\vec{e}_{1f} e^{i(\vec{k}_2 \vec{r})} 
\label{eqFieldLeft} 
\end{equation} 
and for the beam from the right side
\begin{equation} 
\vec{E}_2=A_{2s}\vec{e}_2 e^{-i(\vec{k}_2 \vec{r})} + 
A_{2c}\vec{e}_1^\ast e^{i(\vec{k}_1 \vec{r})} + 
A_{2f}\vec{e}_{2f} e^{i(\vec{k}_1 \vec{r})}. 
\label{eqFieldRight} 
\end{equation} 

\subsection{Equations derivation}

The equations for refractive index gratings can be devided into to sub
classes. The first one is for a crystal with one impurity center:
\begin{eqnarray}  
\left(\frac{\partial}{\partial{t}} 
+\frac{1}{\tau_{eff}}
\left(
1 + i \frac{E_D}{E_q}
\right)
\right){M} 
= 
i B \Gamma \frac{(A_{1s}^\ast A_{1c}+A_{2s}^\ast A_{2c})}{I_0}
+ \frac{M_0}{\tau_{eff}}
, \nonumber \\ 
\left(\frac{\partial}{\partial{t}}
+\frac{1}{\tau_{eff}}
\left(
1 + i \frac{E_D}{E_q}
\right)
\right){M_{1f}} = 
i C_1 \Gamma \frac{ A_{1s}^{\ast} A_{1f}}{I_0}
+ \frac{M_{1f_0}}{\tau_{eff}}
, \nonumber \\ 
\left(\frac{\partial}{\partial{t}} 
+\frac{1}{\tau_{eff}}
\left(
1 + i \frac{E_D}{E_q}
\right)
\right){M_{2f}} = 
i C_2 \Gamma  \frac{A_{2s}^{\ast} A_{2f}}{I_0}
+ \frac{M_{2f_0}}{\tau_{eff}}
, 
\label{eqGrating_1_dpcm} 
\end{eqnarray} 
where
\[
B = \left.const\right|_t,
C_{1,2} = \left.const\right|_t,
M_0 = \left.const\right|_t,
M_{1f_0} = \left.const\right|_t,
M_{2f_0} = \left.const\right|_t.
\] 

For the second case (crystal with 2 impurity centers) we will write
equations that are conjugated to 
(\ref{eqM_2_dpcm}, \ref{eqM1f_2_dpcm}, 
\ref{eqM2f_2_dpcm}):
\begin{eqnarray}
\left(\frac{\partial}{\partial{t}} 
+\frac{1}{\tau_{eff}}
\left(
1 + i \frac{E_D}{E_q}
\right)
\right){M^{(1)}} 
= 
i B \Gamma \frac{(A_{1s}^{\ast} A_{1c}+A_{2s}^{\ast} A_{2c})}{I_0}
-
\nonumber \\
-\frac{1}{\tau_{eff}}
\left(
1 - i \frac{E_D}{E_m^{(1)}}
\right) M^{(2)}
+ \frac{M_0^{(1)}}{\tau_{eff}},
\nonumber \\
\left(
\frac{\partial}{\partial{t}} + \frac{1}{\tau_I^{(2)}}
\right) 
{M^{(2)}} 
= i \frac{\tau_{eff}}{\tau_m^{(2)}}
 B \Gamma^{(2)} \frac{(A_{1s}^{\ast} A_{1c}+A_{2s}^{\ast} A_{2c})}{I_0}
+ \frac{M_0^{(2)}}{\tau_{m}^{(2)}}
,
\nonumber \\
\left(\frac{\partial}{\partial{t}} 
+\frac{1}{\tau_{eff}}
\left(
1 + i \frac{E_D}{E_q}
\right)
\right){M_{1f}^{(1)}} 
= 
i C_1 \Gamma \frac{(A_{1s}^{\ast} A_{1f})}{I_0}
-
\nonumber \\
- \frac{1}{\tau_{eff}}
\left(
1 - i \frac{E_D}{E_m^{(1)}}
\right) M_{1f}^{(2)}
+ \frac{M_{1f_0}^{(1)}}{\tau_{eff}},
\nonumber \\
\left(
\frac{\partial}{\partial{t}} + \frac{1}{\tau_I^{(2)}}
\right) 
{M_{1f}^{(2)}} 
= i \frac{\tau_{eff}}{\tau_m^{(2)}}
C_1 \Gamma^{(2)} \frac{(A_{1s}^{\ast} A_{1f})}{I_0}
+ \frac{M_{1f_0}^{(2)}}{\tau_{m}^{(2)}}
,
\nonumber \\
\left(\frac{\partial}{\partial{t}} 
+\frac{1}{\tau_{eff}}
\left(
1 + i \frac{E_D}{E_q}
\right)
\right){M_{2f}^{(1)}} 
= 
i C_2 \Gamma \frac{(A_{2s}^{\ast} A_{2f})}{I_0}
-
\nonumber \\
- \frac{1}{\tau_{eff}}
\left(
1 - i \frac{E_D}{E_m^{(1)}}
\right) M_{2f}^{(2)}
+ \frac{M_{2f_0}^{(1)}}{\tau_{eff}},
\nonumber \\
\left(
\frac{\partial}{\partial{t}} + \frac{1}{\tau_I^{(2)}}
\right) 
{M_{2f}^{(2)}} 
= i \frac{\tau_{eff}}{\tau_m^{(2)}}
C_2 \Gamma^{(2)} \frac{(A_{2s}^{\ast} A_{2f})}{I_0}
+ \frac{M_{2f_0}^{(2)}}{\tau_{m}^{(2)}}
.
\label{eqGrating_2_dpcm} 
\end{eqnarray}
Where
\[
B = \left.const\right|_t,
C_{1,2} = \left.const\right|_t,
\]
\[
M_0^{(1,2)} = \left.const\right|_t,
M_{1f_0}^{(1,2)} = \left.const\right|_t,
M_{2f_0}^{(1,2)} = \left.const\right|_t.
\] 

The coefficients $B$ and $C_{1,2}$ at (\ref{eqGrating_1_dpcm},
\ref{eqGrating_2_dpcm}) can be understood as overlap ratios. Thus they
define the part of the crystal where interaction occurs. The the
coefficients should  satisfy the following conditions:
$B < 1$ and $C_{1,2} < 1$.

The equations for scattered fields were gotten in our previouse paper
\cite{OurDPCM}. 

For the left light beam's components we got the following result
\begin{eqnarray}
\frac{\partial{A_{1s}}}{\partial{z}} = 
- i D \left\{ 
A_{1c} M^{\ast} + A_{1f} M_{1f}^{\ast} \right\},
\nonumber \\
\frac{\partial{A_{1c}}}{\partial{z}} = - 
i D A_{1s} M,
\nonumber \\
\frac{\partial{A_{1f}}}{\partial{z}} = 
- i D A_{1s} M_{1f},
\label{eqAmpl_left_dpcm}
\end{eqnarray}

For the right light beam's components we got the following result
\begin{eqnarray}
\frac{\partial{A_{2s}}}{\partial{z}} = 
 i D
 \left\{ 
A_{2c} M^{\ast} + A_{2f} M_{2f}^{\ast} \right\},
\nonumber \\
\frac{\partial{A_{2c}}}{\partial{z}} =  
i D
A_{2s} M,
\nonumber \\
\frac{\partial{A_{2f}}}{\partial{z}} = 
i D
A_{2s} M_{2f},
\label{eqAmpl_right_dpcm}
\end{eqnarray}

At (\ref{eqAmpl_right_dpcm}) and (\ref{eqAmpl_left_dpcm}) we assumed
that $\theta_1 =\theta_2=\theta$ thus
$D=\frac{\omega}c\frac 1 {2 \cos \theta}\sqrt{\frac\mu\epsilon}$.

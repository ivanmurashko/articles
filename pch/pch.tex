%% -*- coding:utf-8 -*- 
\documentclass[14pt,a4paper]{article} 

\usepackage[utf8]{inputenc}
\usepackage[english]{babel}
\usepackage{minted}
\usepackage{longtable}
\usepackage{hyperref}
\hypersetup{
  pdftex,
  allcolors=blue,
  bookmarksnumbered=true,     
  bookmarksopen=true,         
  bookmarksopenlevel=1,       
  colorlinks=true,            
  pdfstartview=Fit,           
  pdfpagemode=UseOutlines,  
  pdfpagelayout=TwoPageRight,
  pdftitle={Precompiled headers at clang},
  pdfsubject={Precompiled headers at clang},
  pdfauthor={Ivan Murashkо},
  pdfkeywords={pch, clang, C++}
}
\usepackage{tikz}
\usetikzlibrary{positioning} 
\title{Precompiled headers at clang}
\author{Ivan Murashko}
\date{}
\begin{document}

\maketitle
\tableofcontents

\section*{Introduction}
The doc describes precompiled headers (pch) at clang, their internals and
how they can be used

The source code for examples can be found in the article git
repository \cite{github:articles_ivanmurashko} in the folder 
\textbf{pch/src}.

\section{User guide}
Generate you pch file is simple. Suppose you have a header file with
name \textbf{header.h}:
\inputminted{c++}{./src/simple/header.h} then you can generate a pch for it with
\begin{verbatim}
clang -x c++-header header.h -o header.pch
\end{verbatim}
the option \textbf{-x c++-header} was used there. The option says that
the header file has to be treated as a c++ header file. The output
file is \textbf{header.pch}.

The precompiled header generation is not enough and you may want to
start using them. Typical C++ source file that uses the header may
look like
\inputminted{c++}{./src/simple/main.cpp}
As you may see, the header is included as follows
\begin{minted}{c++}
  ...
#include "header.h"
  ...
\end{minted}
By default clang will not use a pch at the case and you have to
specify it explicitly with
\begin{verbatim}
clang -include-pch header.pch main.cpp -o main -lstdc++
\end{verbatim}


\section{PCH internals}

\bibliographystyle{unsrt}  
\bibliography{pch}     


\end{document}

\documentclass{beamer}

\usepackage{url}
\usepackage{minted}
\usepackage{upquote}
\usepackage{graphicx}
\usepackage{color}
\usepackage{longtable}
\usepackage[nott]{inconsolata}

\usepackage{mathptmx}
\usepackage{helvet}

\definecolor{rcorange}{rgb}{0.99, 0.69, 0.24}
\definecolor{rcblue}{rgb}{0.45, 0.62, 0.81}
\definecolor{tocblue}{rgb}{0.13, 0.29, 0.53}

\mode<presentation>
{
\usetheme{CambridgeUS}
\usecolortheme{seahorse}
\setbeamercolor*{palette primary}{use=structure, fg=black, bg=rcblue}
\setbeamercolor*{palette secondary}{use=structure, fg=black, bg=rcblue}
\setbeamercolor*{palette tertiary}{use=structure, fg=black, bg=rcorange}
\setbeamercolor*{palette quaternary}{use=structure, fg=black, bg=rcorange}
}

%\definecolor{scriptbg}{rgb}{0.95,0.95,0.95}
\definecolor{scriptbg}{rgb}{0.99,0.96,0.90}
\definecolor{hlfg}{HTML}{FE9A2E}
\definecolor{links}{HTML}{2A1B81}
\hypersetup{colorlinks,linkcolor=,urlcolor=links}

\usepackage[T2A]{fontenc}
%\usepackage{lmodern}
\usepackage[utf8x]{inputenc}
\usepackage[english,russian]{babel}
\logo{\includegraphics[height=0.4cm]{RingCentral-logo-256x.png}}
\title[C++ ABI]
    {C++ ABI}
\author{Иван Мурашко}
\date{14 Мая 2018}

\newcommand{\SurrLgt}[1]{\textless{}#1\textgreater{}}
\newcommand{\Hl}[1]{\textcolor{hlfg}{\textbf{#1}}}

\setlength{\parskip}{4pt}

\setbeamercovered{transparent=10}

\setbeamertemplate{section in toc}
{
\begingroup
\color{tocblue}
\inserttocsection
\endgroup
}
\addtobeamertemplate{title page}{}{\tableofcontents}

\begin{document}
\maketitle

\section{API vs ABI}

\begin{frame}[fragile]
\frametitle{Окружение}
\begin{itemize}
\item Centos 7.x, gcc 4.8 (pro)
\item Ubuntu, Fedora Core, gcc 6.x, 7.x, 8.x (dev)
\end{itemize}
\end{frame}

\begin{frame}[fragile]
\frametitle{API}
Application Program Interface
\begin{itemize}
\item Интерфейс
\item Ошибки проявляются на этапе компиляции (Compilation)
\end{itemize}
\end{frame}

\begin{frame}[fragile]
\frametitle{ABI}
\begin{itemize}
\item Двоичное представление, реализация интерфейса
\item Ошибки проявляются на этапе компоновки (Linker)
\end{itemize}
\end{frame}

\section{gcc 5.1 ABI}

\begin{frame}[fragile]
\frametitle{Поддержка стандарта c++11 в gcc}
\begin{itemize}
\item ver. 4.8.1 - объявлена полная поддержка стандарта c++11
\item ver. 5.1 - объявлены серьезные изменения в ABI
\end{itemize}
\end{frame}

\begin{frame}[fragile]
\frametitle{Sizes}
\begin{minted}{c++}
int main() {
  std::vector<int> v;
  std::cout << "std::vector: " << sizeof(v) << std::endl;  
  ...
  std::string str;
  std::cout << "std::string: " << sizeof(str) << std::endl;  
  return 0;
}
\end{minted}
\end{frame}

\begin{frame}[fragile]
\frametitle{Sizes}
\begin{longtable}{|l|c|c|}
\hline
Контейнер STL & gcc 4.9.4 & gcc 5.5.0 \\
\hline
std::vector & 24 & 24 \\
std::queue & 80 & 80 \\
std::priority\_queue & 32 & 32 \\
std::deque & 80 & 80 \\
std::stack & 80 & 80 \\
std::set & 48 & 48 \\
std::multiset & 48 & 48 \\
std::map & 48 & 48 \\
std::multimap & 48 & 48 \\
std::list & 16 & 24 \\
std::string & 8 & 32 \\
\hline
\end{longtable}
\end{frame}

\subsection*{std::list}

\begin{frame}[fragile]
\frametitle{std::list}
\begin{itemize}
\item 16 bytes - двусвязный список
\item Вычисление размера требует $O(N)$ операций
\item Стандарт C++11 требует для этой операции $O(1)$
\end{itemize}
\end{frame}

\subsection*{std::string}

\begin{frame}[fragile]
\frametitle{std::string}
\inputminted{c++}{../src/str.cpp}
\end{frame}

\begin{frame}[fragile]
\frametitle{gcc 5.1 std::string}
\begin{minted}{shell}
$ ./str5
0x7ffd1b434d90
0x7ffd1b434de0
$
\end{minted}
\end{frame}

\begin{frame}[fragile]
\frametitle{gcc 4.9.4 std::string}
\begin{minted}{shell}
$ ./str4
0x127fe88
0x127fe88
$
\end{minted}
\end{frame}


\begin{frame}[fragile]
\frametitle{COW}
\begin{itemize}
\item Преимущества - скорость
\item Недостатки - падение производительности при работе в
  много-поточном окружении
\item COW vs C++11 move
\end{itemize}
\end{frame}


\section{Бинарные внешние библиотеки}

\begin{frame}[fragile]
\frametitle{Изменения в gcc 5.1 API/ABI}
std::list<int> vs 
std::\_\_cxx11::list<int>
\end{frame}

\begin{frame}[fragile]
\frametitle{Ошибки}
\begin{verbatim}
make tests
gcc -g -O0 -c main.cpp 
docker run --rm -v "$PWD":/usr/src/myapp -w /usr/src/myapp gcc:4 \
gcc -shared -fPIC -o libtest4.so -c old.cpp
gcc -g -O0 main.o -o tests -lstdc++ -L. -ltest4
main.o: In function "main":
src/main.cpp:5: undefined reference to 
"test::func(std::__cxx11::basic_string<char, std::char_traits<char>, 
std::allocator<char> >)"
collect2: error: ld returned 1 exit status
Makefile:31: recipe for target "tests" failed
make: *** [tests] Error 1
\end{verbatim}
\end{frame}

\begin{frame}[fragile]
\frametitle{\_GLIBCXX\_USE\_CXX11\_ABI макрос}
\begin{minted}{c++}
#define _GLIBCXX_USE_CXX11_ABI 0
#include <list>
...
// Old ABI be used
std::list<int> l;
...
\end{minted} 
\end{frame}

\begin{frame}[fragile]
\frametitle{\_GLIBCXX\_USE\_CXX11\_ABI макрос по умолчанию}
\begin{minted}{c++}
#define _GLIBCXX_USE_CXX11_ABI 1
#include <list>
...
// New ABI be used
std::list<int> l;
...
\end{minted} 
\end{frame}

\begin{frame}[fragile]
\frametitle{std::abi}
 Herb Sutter, Defining a Portable C++ ABI,

 http://www.open-std.org/Jtc1/sc22/wg21/docs/papers/2014/n4028.pdf
\end{frame}

\section{Примеры}

\subsection*{Использование макроса \_GLIBCXX\_USE\_CXX11\_ABI}


\begin{frame}[fragile]
\frametitle{Интерфейс: ./src/old.h}
\inputminted{c++}{../src/old.h}
\end{frame}

\begin{frame}[fragile]
\frametitle{Реализация: ./src/old.cpp}
\inputminted{c++}{../src/old.cpp}
\end{frame}

\begin{frame}[fragile]
\frametitle{Сборка}
\begin{minted}{makefile}
libtest4.so: old.cpp
    docker run --rm \
               -v "$$PWD":/usr/src/myapp \
               -w /usr/src/myapp gcc:4 \
    gcc -shared -fPIC -o libtest4.so -c old.cpp
\end{minted}
\end{frame}

\begin{frame}[fragile]
\frametitle{Использование библиотеки: ./src/main.cpp}
\inputminted{c++}{../src/main.cpp}
\end{frame}

\begin{frame}[fragile]
\frametitle{Стандартная сборка}
\begin{minted}{makefile}
main.o: main.cpp
	gcc -g -O0 -c main.cpp 

tests: main.o libtest4.so
	gcc -g -O0 main.o -o tests -lstdc++ -L. -ltest4
\end{minted}
\end{frame}

\begin{frame}[fragile]
\frametitle{Стандартная сборка}
\begin{verbatim}
make tests
gcc -g -O0 -c main.cpp 
docker run --rm -v "$PWD":/usr/src/myapp -w /usr/src/myapp gcc:4 \
gcc -shared -fPIC -o libtest4.so -c old.cpp
gcc -g -O0 main.o -o tests -lstdc++ -L. -ltest4
main.o: In function "main":
src/main.cpp:5: undefined reference to 
"test::func(std::__cxx11::basic_string<char, std::char_traits<char>, 
std::allocator<char> >)"
collect2: error: ld returned 1 exit status
Makefile:31: recipe for target "tests" failed
make: *** [tests] Error 1
\end{verbatim}
\end{frame}

\begin{frame}[fragile]
\frametitle{Режим совместимости}
\begin{minted}{makefile}
main.o: main.cpp
	gcc -D_GLIBCXX_USE_CXX11_ABI=0 -g -O0 -c main.cpp

tests: main.o libtest4.so
	gcc -g -O0 main.o -o tests -lstdc++ -L. -ltest4
\end{minted}
\end{frame}

\begin{frame}[fragile]
\frametitle{Режим совместимости}
\begin{verbatim}
make tests
gcc -D_GLIBCXX_USE_CXX11_ABI=0 -g -O0 -c main.cpp 
docker run --rm -v "$PWD":/usr/src/myapp -w /usr/src/myapp gcc:4 \
gcc -shared -fPIC -o libtest4.so -c old.cpp
gcc -g -O0 main.o -o tests -lstdc++ -L. -ltest4
\end{verbatim}
\end{frame}

\subsection*{Адаптер}
\begin{frame}[fragile]
\frametitle{Адаптер: ./src/adapter.h}
\inputminted{c++}{../src/adapter.h}
\end{frame}

\begin{frame}[fragile]
\frametitle{Адаптер: ./src/adapterold.cpp}
\inputminted{c++}{../src/adapterold.cpp}
\end{frame}

\begin{frame}[fragile]
\frametitle{Адаптер: ./src/adapter.cpp}
\inputminted{c++}{../src/adapter.cpp}
\end{frame}

\begin{frame}[fragile]
\frametitle{Сборка}
\begin{minted}{makefile}
adapter.o: adapter.cpp
	gcc -g -O0 -fPIC -o adapter.o -c adapter.cpp

adapterold.o: adapterold.cpp
	gcc -g -O0 -fPIC -o adapterold.o -c adapterold.cpp

libadapter.so: adapter.o adapterold.o
	gcc -shared -fPIC adapter.o adapterold.o -o libadapter.so 

main.o: main.cpp
	gcc -g -O0 -c main.cpp 

tests: main.o libtest4.so libadapter.so
	gcc -g -O0 main.o -o tests -lstdc++ -L. -ladapter -ltest4
\end{minted}
\end{frame}

\section{Заключение}

\begin{frame}[fragile]
\frametitle{Сылки}
\begin{itemize}
\item https://github.com/ivanmurashko
\item articles/cppabi
\end{itemize}
\end{frame}

\begin{frame}[fragile]
\frametitle{Советы}
\begin{enumerate}
\item Пробуйте опции компилятора, такие как
  \texttt{-D\_GLIBCXX\_USE\_CXX11\_ABI=0}
\item Проектируйте использование библиотек через фасады, которые
  скрывают проблемные интерфейсы
\item Ждите реализации std::abi
\end{enumerate}
\end{frame}

\begin{frame}[fragile]
\frametitle{Вопросы}
\begin{center}
\includegraphics[height=5cm]{questions.jpg}
\end{center}
\end{frame}

\end{document}


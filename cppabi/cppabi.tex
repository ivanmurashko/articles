%% -*- coding:utf-8 -*- 
\documentclass[14pt,a4paper]{article} 

\usepackage[utf8]{inputenc}
\usepackage[russian]{babel}
\usepackage{minted}

\title{C++ ABI}
\author{Мурашко И. В.}
\date{}
\begin{document}

\Russian

\maketitle

\section*{Введение}
По роду своей деятельности мы часто сталкиваемся с библиотеками сторонних 
разработчиков которые предоставляются нам в бинарной форме, например 
Oracle OCCI. Вместе с тем в разработке мы используем разные версии 
компиляторов gcc, которые не всегда совместимы по умолчанию с Oracle OCCI. 
В данном докладе детально описываются соответствующие проблемы и способы 
их решения. 

\section{API vs ABI}

TBD

\section{gcc5 ABI}

Версия 5 компилятора gcc была одной из первых, которая полностью (?)
поддерживала новый стандарт c++11. Новые конструкции этого стандарта
тербовали измененений в двоичном представлении некоторых объектов STL
прежде всего это касается std::list, std::string, std::ios\_base,
std::locale. 

Dual ABI
In the GCC 5.1 release libstdc++ introduced a new library ABI that
includes new implementations of std::string and std::list. These
changes were necessary to conform to the 2011 C++ standard which
forbids Copy-On-Write strings and requires lists to keep track of
their size. 

In order to maintain backwards compatibility for existing code linked
to libstdc++ the library's soname has not changed and the old
implementations are still supported in parallel with the new ones.
This is achieved by defining the new implementations in an inline
namespace so they have different names for linkage purposes, e.g. the
new version of std::list<int> is actually defined as
std::__cxx11::list<int>. Because the symbols for the new
implementations have different names the definitions for both versions
can be present in the same library. 

The \_GLIBCXX\_USE\_CXX11\_ABI macro (see Macros) controls whether the
declarations in the library headers use the old or new ABI. So the
decision of which ABI to use can be made separately for each source
file being compiled. Using the default configuration options for GCC
the default value of the macro is 1 which causes the new ABI to be
active, so to use the old ABI you must explicitly define the macro to
0 before including any library headers. (Be aware that some GNU/Linux
distributions configure GCC 5 differently so that the default value of
the macro is 0 and users must define it to 1 to enable the new ABI.) 

Although the changes were made for C++11 conformance, the choice of
ABI to use is independent of the -std option used to compile your
code, i.e. for a given GCC build the default value of the
\_GLIBCXX\_USE\_CXX11\_ABI macro is the same for all dialects. This
ensures that the -std does not change the ABI, so that it is
straightforward to link C++03 and C++11 code together.

Because std::string is used extensively throughout the library a
number of other types are also defined twice, including the
stringstream classes and several facets used by std::locale. The
standard facets which are always installed in a locale may be present
twice, with both ABIs, to ensure that code like
std::use\_facet<std::time_get<char>>(locale); will work correctly for
both std::time\_get and std::\_\_cxx11::time\_get (even if a user-defined
facet that derives from one or other version of time\_get is installed
in the locale). 

Although the standard exception types defined in <stdexcept> use
strings, they are not defined twice, so that a std::out\_of\_range
exception thrown in one file can always be caught by a suitable
handler in another file, even if the two files are compiled with
different ABIs. 

\subsection{std::string}

Рассмотрим следующий фрагмент
\inputminted{c++}{./src/str.cpp}

GCC 5 дает вполне предсказуемый вывод: мы получаем копию строки со
своими внутреннеми данными
\begin{minted}{shell}
$ ./str5
0x7ffc757fc1b0
0x7ffc757fc200
$
\end{minted}

Другая ситуация GCC 4:
\begin{minted}{shell}
$ ./str4
0x1d5be88
0x1d5be88
$
\end{minted}


TBD

\subsection{std::list}

TBD

\subsection{std::ios\_base}

TBD

\section{Бинарные внешние библиотеки}

TBD


\end{document}

%% -*- coding:utf-8 -*- 
\documentclass[14pt,a4paper]{article} 

\usepackage[utf8]{inputenc}
\usepackage[russian]{babel}
\usepackage{minted}
\usepackage{longtable}

\title{C++ ABI}
\author{Мурашко И. В.}
\date{}
\begin{document}

\Russian

\maketitle

\section*{Введение}
По роду своей деятельности мы часто сталкиваемся с библиотеками сторонних 
разработчиков которые предоставляются нам в бинарной форме, например 
Oracle OCCI. Вместе с тем в разработке мы используем разные версии 
компиляторов gcc, которые не всегда совместимы по умолчанию с
используемыми бинарными библиотеками (Oracle OCCI). 
В данном докладе детально описываются соответствующие проблемы и способы 
их решения. Особое внимание уделяется изменениям привнесенным gcc ver.
5.1 в libstdc++.

\section{API vs ABI}

Когда кто-то собирается использовать внешнюю библиотеку, то первое что
интересует - какой интерфейс она предоставляет. Этот интерфейс мы
называем API (Application Program Interface). Кроме этого
интерфейса значение имеет двоично представление данных библиотеки,
которое называется ABI (Application Binary Interface).

API имеет важное значение в момент компиляции исходных кодов, т.е.
ошибки использования API проявляются в момент компиляции.

ABI играет роль при сопряжении уже скомпилированных объектных файлов,
то есть на момент компоновки (link). При этом стоит отметить, что
диагностика ошибок на этапе компиляции более продвинутая чем на этапе
компоновки, что позволяет решать проблемы использования API намного
эффективнее, чем похожие проблемы использования ABI. 


\section{gcc5.1 ABI}

Версия 4.8.1 компилятора gcc была одной из первых, которая полностью
поддерживала новый стандарт c++11. Вместе с тем новые конструкции
этого стандарта требовали изменений в двоичном представлении некоторых
объектов STL, которые были сделаны только в gcc 5.1.

Рассмотрим следующий фрагмент
\inputminted{c++}{./src/sizes.cpp}

Запущенный под разными версиями компилятора gcc он дает следующие
результаты 
\begin{longtable}{|l|c|c|}
\hline
Контейнер STL & gcc 4.9.4 & gcc 5.5.0 \\
\hline
std::vector & 24 & 24 \\
std::queue & 80 & 80 \\
std::priority\_queue & 32 & 32 \\
std::deque & 80 & 80 \\
std::stack & 80 & 80 \\
std::set & 48 & 48 \\
std::multiset & 48 & 48 \\
std::map & 48 & 48 \\
std::multimap & 48 & 48 \\
std::list & 16 & 24 \\
std::string & 8 & 32 \\
\hline
\end{longtable}

Как видно двоичное представление как минимум двух контейнеров
(std::list и std::string) различается между этими двумя версиями
компиляторов.

Связано это прежде всего с тем, что libstdc++ GCC 5.1 ввела
\cite{DualABI}  новую имплементацию данных контейнеров. Если
посмотреть на размер std::list в gcc 4 то видно что размер равен
удвоенному размеру
указателя (16 bytes), что соответствует представлению std::list как
двусвязанного списка: каждый элемент содержит указатели на предыдущий
и следующий элементы. Такой способ хранения не соответствует стандарту
C++11, который явно требует чтобы вычисление размера контейнеров, и
в частности списка имело бы сложность $O(1)$ \cite{ISO:2012:III}, что требует хранения
дополнительных данных вместе с указателями на предыдущий и последующий
элементы.   


%% прежде всего это касается std::list, std::string, std::ios\_base,
%% std::locale. 

%% Dual ABI
%% In the GCC 5.1 release libstdc++ introduced a new library ABI that
%% includes new implementations of std::string and std::list. These
%% changes were necessary to conform to the 2011 C++ standard which
%% forbids Copy-On-Write strings and requires lists to keep track of
%% their size. 

%% In order to maintain backwards compatibility for existing code linked
%% to libstdc++ the library's soname has not changed and the old
%% implementations are still supported in parallel with the new ones.
%% This is achieved by defining the new implementations in an inline
%% namespace so they have different names for linkage purposes, e.g. the
%% new version of std::list<int> is actually defined as
%% std::__cxx11::list<int>. Because the symbols for the new
%% implementations have different names the definitions for both versions
%% can be present in the same library. 

%% The \_GLIBCXX\_USE\_CXX11\_ABI macro (see Macros) controls whether the
%% declarations in the library headers use the old or new ABI. So the
%% decision of which ABI to use can be made separately for each source
%% file being compiled. Using the default configuration options for GCC
%% the default value of the macro is 1 which causes the new ABI to be
%% active, so to use the old ABI you must explicitly define the macro to
%% 0 before including any library headers. (Be aware that some GNU/Linux
%% distributions configure GCC 5 differently so that the default value of
%% the macro is 0 and users must define it to 1 to enable the new ABI.) 

%% Although the changes were made for C++11 conformance, the choice of
%% ABI to use is independent of the -std option used to compile your
%% code, i.e. for a given GCC build the default value of the
%% \_GLIBCXX\_USE\_CXX11\_ABI macro is the same for all dialects. This
%% ensures that the -std does not change the ABI, so that it is
%% straightforward to link C++03 and C++11 code together.

%% Because std::string is used extensively throughout the library a
%% number of other types are also defined twice, including the
%% stringstream classes and several facets used by std::locale. The
%% standard facets which are always installed in a locale may be present
%% twice, with both ABIs, to ensure that code like
%% std::use\_facet<std::time_get<char>>(locale); will work correctly for
%% both std::time\_get and std::\_\_cxx11::time\_get (even if a user-defined
%% facet that derives from one or other version of time\_get is installed
%% in the locale). 

%% Although the standard exception types defined in <stdexcept> use
%% strings, they are not defined twice, so that a std::out\_of\_range
%% exception thrown in one file can always be caught by a suitable
%% handler in another file, even if the two files are compiled with
%% different ABIs. 



\subsection{std::string}

За изменением размера std::string скрываются более глобальные
изменения нежели те, что были сделаны в std::list. 
Рассмотрим следующий фрагмент
\inputminted{c++}{./src/str.cpp}

GCC 5 дает вполне предсказуемый вывод: мы получаем копию строки со
своими внутренними данными
\begin{minted}{shell}
$ ./str5
0x7ffeb1357230
0x7ffeb1357280
$
\end{minted}

Другая ситуация GCC 4:
\begin{minted}{shell}
$ ./str4
0x1e13e88
0x1e13e88
$
\end{minted}

Как видно несмотря на то, что мы взяли копию объекта, реально данные
скопированы не были. Связано это с тем, что GCC 4 использует COW
(copy-on-write) методологию 
для имплементации строк и в, частности счетчик (reference count) для
отслеживания ссылок на используемые объекты. Таким образом, если объект
копируется, то увеличивается счетчик ссылок. Реальное копирование
объекта с выделением памяти происходит только при изменении данных.
Такая реализация была 
лишь одной из возможных, и не смотря на плюсы (скорость работы), имела
и свои минусы связанные прежде всего с много-поточностью (необходимая
синхронизация убивает все плюсы COW), в частности MS VS 2005
использовала другую методику, которая заключалась в реальном
копировании данных. 

Стандарт C++11 ввел move семантику, которая устранила основные
минусы обычной реализации строк, с выделением памяти в момент
создания. Таким образом в подавляющем большинстве случаев
использование COW в данный момент неоправданно, что привело к прямому
запрету использования COW для строк в стандарте C++11
\cite{ISO:2012:III}. 

\section{Бинарные внешние библиотеки}

В идеальном варианте внешняя библиотека поставляется в виде исходников
по какой-то из свободных лицензий (GPL, MIT, etc.). В этом случае на
целевой платформе может быть собрана бинарная версия библиотеки,
которая будет совместима со всеми другими библиотеками собранными
похожим образом. 

В проблематичном случае бинарная библиотека собирается на некоторой
платформе которая отличается от целевой, например Oracle OCCI (???)
собирается с помощью компилятора gcc 4 и рассчитана на применение в
системах RedHat Linux 7. В случае если целевая платформа подразумевает
использование компилятора gcc 5.x вместе с другими библиотеками
собранными этим компилятором (например boost) то возникнут проблемы
на этапе компоновки, которые выглядят следующим образом

\begin{verbatim}
make tests
gcc -g -O0 -c main.cpp 
docker run --rm -v "$PWD":/usr/src/myapp -w /usr/src/myapp gcc:4 \
gcc -shared -fPIC -o libtest4.so -c old.cpp
gcc -g -O0 main.o -o tests -lstdc++ -L. -ltest4
main.o: In function "main":
src/main.cpp:5: undefined reference to 
"test::func(std::__cxx11::basic_string<char, std::char_traits<char>, 
std::allocator<char> >)"
collect2: error: ld returned 1 exit status
Makefile:31: recipe for target "tests" failed
make: *** [tests] Error 1
\end{verbatim}

Для решения этих проблем gcc ABI предлагает реализацию контейнеров
std::list и std::string
удовлетворяющую требованиям стандарта c++11 
 в отдельном namespace, т.е. std::list<int> реально будет 
std::\_\_cxx11::list<int>. Что позволяет использовать две версии ABI в
одной программе. На уровне исходников выбор версии ABI выбирается с помощью
следующего макроса
\begin{minted}{c++}
#define _GLIBCXX_USE_CXX11_ABI 0
#include <list>
...
// Old ABI be used
std::list<int> l;
...
\end{minted} 

По умолчанию используется (неявно) следующий вариант
\begin{minted}{c++}
#define _GLIBCXX_USE_CXX11_ABI 1
#include <list>
...
// New ABI be used
std::list<int> l;
...
\end{minted} 

Таким образом, если проекту требуется только одна внешняя библиотека с
несовместимым интерфейсом которая недоступна в исходных кодах, то все
остальные библиотеки, как и исходный код проекта должны быть собраны в
режиме совместимости со старым ABI, для чего следующий флаг должен быть
добавлен в опции компилятора gcc:\texttt{-D\_GLIBCXX\_USE\_CXX11\_ABI=0}. 

Сложнее ситуация когда имеется несколько несовместимых бинарных
библиотек которые должны быть использованы в рамках одного проекта.
В этом случае возможны варианты. Herb Sutter предложил \cite{sutter}
расширение языковых конструкций языка C++ в котором предполагается
иметь две версии STL. Одна стандартная, которая доступна под
стандартным namespace std, например std::string. В этой версии
библиотеки хранятся самые последние версии реализации STL. При этом
также существует стабильная версия, под namespace std::abi, например
std::abi::string которая является переносимой с точки зрения бинарного
интерфейса. Авторы внешних библиотек предлагают вариант своей
библиотеки в режиме std::abi, что позволяет ее использовать с
различными версиями компиляторов.

До тех пор пока предложение \cite{sutter} еще не вступило в силу имеет
смысл рассмотреть организацию взаимодействия с внешними библиотеками с
помощью адаптеров, как это сделано в примерах ниже.

\section{Примеры}

Рассмотрим тестовую библиотеку, которая состоит из интерфейса (API):
old.h
\inputminted{c++}{./src/old.h}
и реализации 
\inputminted{c++}{./src/old.cpp}
сборка осуществляется с помощью следующей команды
\begin{minted}{makefile}
libtest4.so: old.cpp
	docker run --rm -v "$$PWD":/usr/src/myapp -w /usr/src/myapp gcc:4 \
	gcc -shared -fPIC -o libtest4.so -c old.cpp
\end{minted}

Код который использует эту библиотеку выглядит следующим образом: main.cpp
\inputminted{c++}{./src/main.cpp}

Если использовать стандартные опции сборки
\begin{minted}{makefile}
main.o: main.cpp
	gcc -g -O0 -c main.cpp 

tests: main.o libtest4.so
	gcc -g -O0 main.o -o tests -lstdc++ -L. -ltest4
\end{minted}
то получится следующее сообщение об ошибке
\begin{verbatim}
make tests
gcc -g -O0 -c main.cpp 
docker run --rm -v "$PWD":/usr/src/myapp -w /usr/src/myapp gcc:4 \
gcc -shared -fPIC -o libtest4.so -c old.cpp
gcc -g -O0 main.o -o tests -lstdc++ -L. -ltest4
main.o: In function "main":
src/main.cpp:5: undefined reference to 
"test::func(std::__cxx11::basic_string<char, std::char_traits<char>, 
std::allocator<char> >)"
collect2: error: ld returned 1 exit status
Makefile:31: recipe for target "tests" failed
make: *** [tests] Error 1
\end{verbatim}


Если перевести весь проект на старый ABI, т.е. использовать для сборки
флаг \texttt{-D\_GLIBCXX\_USE\_CXX11\_ABI=0}, то Makefile будет выглядеть
следующим образом
\begin{minted}{makefile}
main.o: main.cpp
	gcc -D_GLIBCXX_USE_CXX11_ABI=0 -g -O0 -c main.cpp

tests: main.o libtest4.so
	gcc -g -O0 main.o -o tests -lstdc++ -L. -ltest4
\end{minted}
и весь проект соберется успешно
\begin{verbatim}
make tests
gcc -D_GLIBCXX_USE_CXX11_ABI=0 -g -O0 -c main.cpp 
docker run --rm -v "$PWD":/usr/src/myapp -w /usr/src/myapp gcc:4 \
gcc -shared -fPIC -o libtest4.so -c old.cpp
gcc -g -O0 main.o -o tests -lstdc++ -L. -ltest4
\end{verbatim}

В качестве альтернативного варианта разработаем адаптер состоящий из
следующих файлов - adapter.h:
\inputminted{c++}{./src/adapter.h}
adapterold.cpp:
\inputminted{c++}{./src/adapterold.cpp}
adapter.cpp:
\inputminted{c++}{./src/adapter.cpp}

Сборка теперь осуществляется с помощью следующего Makefile:
\begin{minted}{makefile}
adapter.o: adapter.cpp
	gcc -g -O0 -fPIC -o adapter.o -c adapter.cpp

adapterold.o: adapterold.cpp
	gcc -g -O0 -fPIC -o adapterold.o -c adapterold.cpp

libadapter.so: adapter.o adapterold.o
	gcc -shared -fPIC adapter.o adapterold.o -o libadapter.so 

main.o: main.cpp
	gcc -g -O0 -c main.cpp 

tests: main.o libtest4.so libadapter.so
	gcc -g -O0 main.o -o tests -lstdc++ -L. -ladapter -ltest4
\end{minted}

\section*{Заключение}
Несколько полезных советов по результатам этого исследования. 
\begin{enumerate}
\item Пробуйте опции компилятора, такие как
  \texttt{-D\_GLIBCXX\_USE\_CXX11\_ABI=0}
\item Проектируйте использование библиотек через фасады, которые
  скрывают проблемные интерфейсы
\item Ждите реализации std::abi \cite{sutter}
\end{enumerate}

Ну, и на конец самый главный совет: избегайте использования библиотек,
которые недоступны в исходных кодах.

\bibliographystyle{gost780s}  %% стилевой файл для оформления по
\bibliography{cppabi}     %% имя библиографической базы (bib-файла)


\end{document}

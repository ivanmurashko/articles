%% -*- coding:utf-8 -*-
\chapter{Natural transformation}

Natural transformation is the most important part of the category
theory. It provides a possibility to compare \mynameref{def:functor}s
via a standard tool. 

\section{Definitions}

The natural transformation is not an easy concept compare other one
and requires some additional preparations before we can give the
formal definition.

\begin{figure}
  \centering
  \begin{tikzpicture}[ele/.style={fill=black,circle,minimum
        width=.8pt,inner sep=1pt},every fit/.style={ellipse,draw,inner
        sep=-2pt}]

    % the texts

    \node at (0,3) {$C$};        
    \node at (4,3) {$D$};        

    \node[ele,label=above:$a$] (a) at (0,2) {};    
    \node[ele,label=above:$a_F$] (af) at (4,2) {};
    \node[ele,label=below:$a_G$] (ag) at (4,0) {};

    \node[draw,fit= (a),minimum width=2cm, minimum
      height=3.5cm] {} ;
    \node[draw,fit= (af) (ag),minimum width=2cm, minimum
      height=3.5cm] {} ;

    \draw[->,thick,shorten <=2pt,shorten >=2pt] (a) to
    node[above]{$F$} (af);
    \draw[->,thick,shorten <=2pt,shorten >=2pt] (a) to
    node[above]{$G$} (ag);
    \draw[->,thick,shorten <=2pt,shorten >=2pt] (af) to
    node[right]{$\alpha_a$} (ag);
  \end{tikzpicture}
  \caption{Natural transformation object mapping}
  \label{fig:nt_objects_mapping}
\end{figure}

Consider 2 categories $\cat{C}, \cat{D}$ and 2
\mynameref{def:functor}s $F: \cat{C} \to \cat{D}$ and $G: \cat{C} \to
\cat{D}$. If we have an \mynameref{def:object} $a \in \catob{C}$ then
it will be translated by different functors into different objects of
category $\cat{D}$: $a_F = F a, a_G = G a \in \catob{D}$ (see
\cref{fig:nt_objects_mapping}). There are 2 options possible
\begin{enumerate}
\item There is not  any \mynameref{def:morphism} that connects $a_F$
  and $a_G$.
\item $\exists \alpha_a \in \hom\left(a_F, a_G\right)$.
\end{enumerate}
We can of course to create an artificial morphism that connects the
objects but if we use \textit{natural} morphisms 
\footnote{the word natural means that already existent morphisms from
  category $\cat{D}$ are used}
then we can get a
special characteristic of the considered functors and categories. For
instance if we have a lot of such morphisms then we can say that the
considered functors are related each other. Opposite example if there
is no such morphisms then the functors can be considered as unrelated
each other. Another example if the
morphisms are \mynameref{def:isomorphism}s then the functors can be
considered as equal.

\begin{definition}[Natural transformation]
\label{def:nt}
TBD
\end{definition}

TBD

\section{Examples}

\subsection{\textbf{Set} category}
TBD

\subsection{Programming languages}

\begin{theorem}[Reynolds]
\label{thm:reynolds}
 parametricity theorem 
TBD
\end{theorem}

\subsubsection{\textbf{Hask} category}
TBD

%% -*- coding:utf-8 -*-
\chapter{Base definitions}

\section{Definitions}

\subsection{Object}

\begin{definition}[Class]
  A class is a collection of sets (or sometimes other mathematical
  objects) that can be unambiguously defined by a property that all
  its members share. 
  \label{def:class}
\end{definition}

\begin{definition}[Object]
\label{def:object}
In category theory object is considered as something that does not
have internal structure (aka point) but has a property that makes
different objects belong to the same \mynameref{def:class}
\end{definition}

\begin{remark}[Class of Objects]
\label{rem:objclass}
The \mynameref{def:class} of \mynameref{def:object}s will be marked as 
$\mathrm{ob}(C)$
\end{remark}

\subsection{Morphism}
Morphism is a kind of relation between 2 \mynameref{def:object}s. 
\begin{definition}[Morphism]
\label{def:morphism}
A relation between two \mynameref{def:object}s $a$ and $b$ 
\[
f_{ab}: a \rightarrow b
\]
is called
\textit{morphism}. Morphism assumes a direction i.e. one \mynameref{def:object}
($a$) is called \textit{source} and another one ($b$) \textit{target}.
\end{definition}

\mynameref{def:morphism}s have several properties. \footnote{The
  properties don't have any proof and postulated as axioms}
\begin{property}[Composition]
\label{prop:composition}
If we have 3 \mynameref{def:object}s $a, b$ and $c$ and 2
\mynameref{def:morphism}s 
\[
f_{ab} : a \rightarrow b
\]
and 
\[
f_{bc} : b \rightarrow c
\]
then there exists \mynameref{def:morphism} 
\[
f_{ac} : a \rightarrow c
\]
such that
\[
f_{ac} = f_{bc} \circ f_{ab}
\]
\end{property}

\begin{remark}[Composition]
\label{rem:composition}
The equation
\[
f_{ac} = f_{bc} \circ f_{ab}
\]
means that we apply $f_{ab}$ first and then we apply $f_{bc}$ to the
result of the application i.e. if our objects are sets and $x \in a$
then 
\[
f_{ac} ( x ) = f_{bc} ( f_{ab} ( x ) ),
\]
where $f_{ab} ( x ) \in b$.
\end{remark}

\begin{property}[Associativity]
\label{prop:associativity}
The \mynameref{def:morphism}s \mynameref{prop:composition}s should
follow associativity property:
\[
f_{ce} \circ (f_{bc} \circ f_{ab}) = (f_{ce} \circ f_{bc}) \circ
f_{ab} = f_{ce} \circ f_{bc} \circ f_{ab}.
\]
\end{property}

\begin{definition}[Identity morphism]
\label{def:id}
For every \mynameref{def:object} $a$ we define a special
\mynameref{def:morphism} $\idarrow[a] : a \rightarrow a$ with the
following properties: $\forall f_{ab} : a \rightarrow b$
\[
\idarrow[a] \circ f_{ab} = f_{ab}
\]
and
$\forall f_{ba} : b \rightarrow a$
\[
f_{ba} \circ \idarrow[a]  = f_{ba}.
\] 
This morphism is called \textit{identity morphism}.
\end{definition}

\begin{definition}[Commutative diagram]
  A commutative diagram is a diagram of \mynameref{def:object}s (also known as
  vertices) and \mynameref{def:morphism}s (also known as arrows or
  edges) such that all directed paths in the diagram with the same
  start and endpoints lead to the same result by composition
  \label{def:commutativediagram}

  The following diagram commutes if $f_{ab} = f_{cb} \circ f_{ac}$.

  \begin{tikzpicture}[description/.style={fill=white,inner sep=2pt}]
    \matrix (m) [matrix of math nodes, row sep=3em,
      column sep=2.5em, text height=1.5ex, text depth=0.25ex]
            { A& & B \\
              & C & \\ };
            %\draw[double,double distance=5pt] (m-1-1) – (m-1-3);
            \path[->]
            (m-1-1) edge node[description] {$ f_{ab} $} (m-1-3)
            edge node[description] {$  f_{ac} $} (m-2-2)
            (m-2-2) edge node[description] {$  f_{cb} $} (m-1-3);
  \end{tikzpicture}
\end{definition}


\begin{remark}[Class of Morphisms]
\label{rem:morphclass}
The \mynameref{def:class} of \mynameref{def:morphism}s will be marked as 
$\mathrm{hom}(C)$
\end{remark}

\begin{definition}[Monomorphism]
\label{def:monomorphism}
If $\forall g_1, g_2$ the equation 
\[
f \circ g_1 = f \circ g_2
\]
leads to 
\[
g_1 = g_2
\]
then $f$ is called \textit{monomorphism}.
\end{definition}

\begin{definition}[Epimorphism]
\label{def:epimorphism}
If $\forall g_1, g_2$ the equation 
\[
g_1 \circ f = g_2 \circ f
\]
leads to 
\[
g_1 = g_2
\]
then $f$ is called \textit{epimorphism}.
\end{definition}


\subsection{Category}

\begin{definition}[Category]
\label{def:category}
A category $\cat{C}$ consists of 
\begin{itemize}
\item \mynameref{def:class} of
\mynameref{def:object}s $\mathrm{ob}(C)$
\item \mynameref{def:class} of \mynameref{def:morphism}s $\mathrm{hom}(C)$
defined for $\mathrm{ob}(C)$, i.e. each morphism $f_{ab}$ from 
$\mathrm{hom}(C)$ has both source
$a$ and target $b$ from $\mathrm{ob}(C)$
\end{itemize}
\end{definition}

\subsubsection{Equality of objects}
via unique isomorphism

\subsubsection{Equality of morphisms}
TBD

\section{Examples}

There are several examples of categories that will also be used later

\subsection{\textbf{Set} category}
\begin{example}[{Set} category]
\label{ex:setcategory}
\index{Object!\textbf{Set} example}
\index{Morphism!\textbf{Set} example}
In the set category we consider a set of sets where
\mynameref{def:object}s are sets and \mynameref{def:morphism}s are
functions between the sets. 
\end{example}

\begin{remark}[Set vs Category]
\label{rem:set_vs_category}
There is an interesting relation between sets and categories. In both
we consider objects(sets) and relations between
them(morphisms/functions). 

In the set theory we can get info about functions by looking inside
the objects(sets) aka use ``microscope'' \cite{bib:milewski2018category} 

Contrary in the category theory we initially don't have info about object
internal structure but can get it using the relation between the
objects i.e. using \mynameref{def:morphism}s. In other words we can use
``telescope'' \cite{bib:milewski2018category}  there.
\end{remark}

\begin{definition}[Surjection]
  \label{def:surjection}
  The function $f: X \rightarrow Y$ is surjective (or onto) if
  $\forall y \in Y$, $\exists x \in X$ such that
  $f\left(x\right) = y$.
\end{definition}

\begin{remark}[Surjection vs Epimorphism]
\label{rem:surjection_epimorphism}
TBD
\end{remark}

\begin{definition}[Injection]
  \label{def:injection}
  The function $f: X \rightarrow Y$ is injective (or one-to-one function) if
  $\forall x_1, x_2 \in X$, such that $x_1 \ne x_2$ then
  $f\left(x_1\right) \ne f\left(x_2\right)$.
\end{definition}

\begin{remark}[Injection vs Monomorphism]
\label{rem:injection_monomorphism}
TBD
\end{remark}


\subsection{\textbf{Hask} category}
TBD

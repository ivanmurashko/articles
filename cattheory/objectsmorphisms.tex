%% -*- coding:utf-8 -*-
\chapter{Objects and morphisms}

\section{Equality}

The important question is how can we decide whenever an
object/morphism is equal to another object/morphism? The trivial
answer is possible for if an \mynameref{def:object} is a
\mynameref{def:set}. In this case we can say that 2 objects are equal
if they contains the same elements. Unfortunately we cannot do the
same for default objects as soon as they don't have any internal
structure. We can use the same trick as in
\mynameref{rem:set_vs_category}: if we cannot use ``microscope'' lets
use ``telescope'' and define the equality of objects and morphisms of
a category $c$ in the terms of whole $\mathrm{hom}(C)$.

\begin{definition}[Objects equality]
\label{def:object_equality}
Two \mynameref{def:object}s $a$ and $b$ in \mynameref{def:category}
$C$ are equal if there exists an unique \mynameref{def:isomorphism}
$f: a \to b$. This also means that also exist unique isomorphism $g: b
-> a$. These two \mynameref{def:morphism}s are related each other via
the following equations: $f \circ g = \idarrow[a]$ and $g \circ f
= \idarrow[b]$. 
\end{definition}

Unlike \mynameref{def:function}s between \mynameref{def:set}s we don't
have any additional info 
\footnote{
for instance info about sets internals. i.e. which elements of the sets
are connected by the considered functions
}
about \mynameref{def:morphism}s except
category theory axioms which the morphisms satisfied
\cite{bib:stackexchange:morphism:equality}. This leads us to the
following definition for morphims equality:
\begin{definition}[Morphisms equality]
\label{def:morphism_equality}
Two \mynameref{def:morphism}s $f$ and $g$ in \mynameref{def:category}
$C$ are equal if the equality can be derived from base axioms: 
\begin{itemize}
\item \mynameref{prop:composition}
\item \mynameref{prop:associativity}
\item \mynameref{def:id}: \eqref{eq:leftid}, \eqref{eq:rightid}
\end{itemize}
or \mynameref{def:commutative_diagram}s which postulates the equality.
\end{definition}

\section{Initial and terminal objects}
TBD

\section{Product and sum}
TBD

\section{Examples}

\subsection{\textbf{Set} category}
TBD

\subsection{Programming languages}
\subsubsection{\textbf{Hask} category}
TBD
\subsubsection{\textbf{C++} category}
TBD



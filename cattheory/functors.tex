%% -*- coding:utf-8 -*-
\chapter{Functors}

\section{Definitions}

\begin{definition}[Functor]
\label{def:functor}
Let $\cat{C}$ and $\cat{D}$ are 2 categories. A mapping $F: \cat{C}
\to \cat{D}$ between the categories is called \textit{functor} is it
preserves the internal structure: 
\begin{itemize}
\item $\forall a_C \in \catob{C}, \exists a_D \in \catob{D}$ such that
  $a_d = F( a_C )$
\item $\forall f_C \in \cathom{C}, \exists f_D \in \cathom{D}$ such
  that $\dom f_D = F (\dom f_C), \cod f_D = F (\cod f_C)$. We will use
  the following notation later: $f_D = F(f_C)$.
\item $\forall f_C, g_C$ the following equation holds: 
\[
F\left(f_C \circ
  f_D\right) = F\left(f_C\right) \circ F\left(g_C\right) = f_D \circ
  g_D.
\]
\end{itemize}  
\end{definition}

\section{Natural transformations}

TBD

\section{Examples}

\subsection{\textbf{Set} category}
TBD

\subsection{Programming languages}

\subsubsection{\textbf{Hask} category}
TBD

\subsubsection{\textbf{C++} category}
TBD

\subsubsection{\textbf{Scala} category}
TBD

\subsection{Quantum mechanics}
TBD

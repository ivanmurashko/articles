%% -*- coding:utf-8 -*-
\chapter*{Introduction}

You just looked at yet another introduction to Category Theory. The
subject mostly consists of a lot of definitions that are related each
others and I wrote the book to collect all of them in one
place to be easy checked and updated in future when I decide to refresh
my knowledge about the field of math. Therefore the book was written mostly
for my category theory studying 
purposes but I will appreciate if somebody else find it useful.

The topics(chapters) cover the base definitions
(\mynameref{def:object}, \mynameref{def:morphism} and
\mynameref{def:category}), \mynameref{def:functor},
\mynameref{def:nt}, \mynameref{def:monad} and also include important
results from the category theory such as Yoneda's lemma (see
\cref{sec:yoneda}) and Curry-Howard-Lambek correspondence (see
\cref{sec:curry_howard_lambek}).  

There are a lot of 
examples in each chapter. The examples cover different category
theory application areas. I assume that the reader is familiar with
the corresponding area and the example(s) can be passed if not. I.e.
anyone can choose the suitable example(s) for (s)he. 

The most important examples are related to the set theory. The set
theory and category theory are very close related. Each one can be
considered as an alternative view to another one.

There are a lot of examples from programming languages which include
Haskell, Scala, C++. The source files for programming languages 
examples (Haskell, C++, Scala) can be found on github repository
\cite{bib:github:ivanmurashko}.  

The examples from physics are related to quantum mechanics that is the
most known for me. For the examples I am inspired by the Bob Coecke
article \cite{bib:arxiv:Bob_Coecke_2008}.

The text is distributed under \textbf{Creative Common Public License}
(see the text of the license at the end of the book)
i.e. any reader has right to copy, store, modify, distribute or build
upon the book until the original author is pointed in the derivative
products. 

%% -*- coding:utf-8 -*-
\chapter{Neural network basics}

\section{Logistic regression}
Lets consider a simple example of logistic regression that can be used for
classification task. I.e. it will provide \textbf{yes} or \textbf{no} answer on
a question about input data.

Suppose our input data are represented as the following vector
\[
\vec{x} =
\begin{bmatrix}
  x_{1} \\
  x_{2} \\
  \vdots \\
  x_{m}
\end{bmatrix}
\]
The result is a scalar \(y \in \{1,0\}\)

In the model of supervised learning we have a set of $m$ samples that are used
for learning process:
\[
\begin{array}[c]
  \vec{x}^{(1)} \rightarrow y^{(1)}, \\
  \vec{x}^{(2)} \rightarrow y^{(2)}, \\
  \vdots \\
  \vec{x}^{(m)} \rightarrow y^{(m)}, \\
\end{array}
\]
We want to construct a function $\mathcal{F}$ such that
\[
\mathcal{F}\left(\vec{x}^{(i)}\right) = y^{(i)}
\]
for all $i \in \{1, \dots, m\}$

TBD

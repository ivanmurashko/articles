%% -*- coding:utf-8 -*- 
\documentclass[14pt,a4paper]{article} 
\usepackage[T1,T2A]{fontenc}
\usepackage[utf8]{inputenc}
\usepackage{minted}
\usepackage{longtable}
\usepackage{hyperref}
\hypersetup{
  pdftex,
  allcolors=blue,
  bookmarksnumbered=true,     
  bookmarksopen=true,         
  bookmarksopenlevel=1,       
  colorlinks=true,            
  pdfstartview=Fit,           
  pdfpagemode=UseOutlines,  
  pdfpagelayout=TwoPageRight,
  pdftitle={Probability paradoxes},
  pdfsubject={Probability},
  pdfauthor={Ivan Murashkо},
  pdfkeywords={probability, paradox}
}
\usepackage{tikz}
\usetikzlibrary{positioning} 
\title{Probability paradoxes}
\author{Ivan Murashko}
\date{}
\begin{document}

\maketitle
\tableofcontents

\section*{Introduction}
The goal for the article is to demonstrate several paradoxes that are
related to probability theory and how can they can be solved.

\section{Base definitions of probability theory}
TBD \cite{bib:kolmogorov74basic}

\section{Monty Hall problem}
TBD

\section{Waiting time on a bus stop}
TBD

\bibliographystyle{gost780s}  
\bibliography{probability}     


\end{document}

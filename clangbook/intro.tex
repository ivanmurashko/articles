%% -*- coding:utf-8 -*-
\chapter*{Introduction}

The \mintinline{text}{clang} is C/C++ and ObjC compiler. It's an
integral part of LLVM 
project. When we say about clang we can say about 2 different
things. The first one is the \myindex{compiler frontend} i.e. the part of
compiler that is responsible for parsing and semantic reasoning about
the program. We also use the word \clang when we say
about the compiler itself. It's also referred as compiler driver. The
driver is responsible for compiler invocation i.e. it can be
considered as a manager that calls different parts of the
compiler such as the compiler
frontend as well as other parts that are required for successful
compilation (middle-end, back-end, assembler, linker).

The book is mostly focused on the \clang compiler
frontend but it also includes some other relevant parts of
LLVM that are critical for the frontend internals. The LLVM project evolves
very fast and some of its parts might be completely rewritten between different
revisions. We will use a specific version of LLVM at the book. The used version
is 15.x. It was first released in September 2022
\cite{llvm:releases} \label{c:release15}.  

The book is separated into 2 parts. The first one provides basic
info about LLVM project and how it can be installed. It also describes
useful development tools and configurations used for LLVM code
exploration later in the book. The internal \clang architecture is the next
primary topic for the first part of the book. The knowledge about \clang
internals and its place inside LLVM is essential for any development related to
\clang. The final topic for the first part is compilation performance and
especially how it can be improved. You can find a description for several \clang
features that might significantly improve the compilation speed. There are C++ modules, header maps and others.

The \clang follows primary paradigm of LLVM -
everything is a library, that allows to create a bunch of different
tools. The second part of the book is about such tools. We consider clang-tidy -
powerful framework to create lint checks. We consider simple checks based on AST
(abstract syntax tree) matching as well as more powerful ones based on advanced
techniques such as CFG (control flow graph). The list of tools is not limited by
the code analysis but also include refactoring tools as well as IDE support.


The book also includes a lot of different examples that can be found at
\cite{github:clangbook_src}.

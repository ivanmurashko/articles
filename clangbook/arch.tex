%% -*- coding:utf-8 -*-
\chapter{Architecture}

You can find some info about clang internal architecture and relation
with other LLVM components.

When we spoke about \textbf{clang} we should separate 2 things:
\begin{itemize}
\item driver
\item compiler frontend 
\end{itemize}
Both of them are called \textbf{clang} but perform different
operations. The driver invokes different stages of compilation process
(see fig. \ref{fig:clang_driver}). 
\begin{figure}
\begin{center}
\smartdiagramset{border color=none,
back arrow disabled=true}
\smartdiagram[flow diagram:horizontal]{Frontend,Middle-end, Backend,Assembler,Linker}
\end{center}
  \caption{Clang driver}
  \label{fig:clang_driver}
\end{figure}
The stages are standard for ordinary compiler and nothing special is
there:
\begin{itemize}
\item Frontend: it does lexical analysis and parsing.
\item Middle-end: it does different optimization on the intermediate
  representation (LLVM-IR) code
\item Backend: Native code generation
\item Assembler: Running assembler
\item Linker: Running linker
\end{itemize}

The driver is invoked by the following command
\begin{minted}{text}
clang main.cpp -o main -lstdc++
\end{minted}
The driver also adds a lot of additional arguments, for instance
search paths for system includes that could be platform specific.
You can use \mintinline{text}{-###} clang option to print actual
command line used by the driver
\begin{minted}[breaklines]{text}
$clang -### main.cpp -o main -lstdc++
clang version 12.0.1 (Fedora 12.0.1-1.fc34)
Target: x86_64-unknown-linux-gnu
Thread model: posix
InstalledDir: /usr/bin
 "/usr/bin/clang-12" "-cc1" "-triple" "x86_64-unknown-linux-gnu" "-emit-obj" "-mrelax-all" ...
\end{minted}

One may see that the clang compiler toolchain corresponds the pattern
wildly described at different compiler books
\cite{book:engineering_a_compiler}. Despite the fact, the frontend
part is quite different from a typical compiler frontend described at
the books. The primary reason for this is C++ language and it's
complexity, for instance macros. 

%% -*- coding:utf-8 -*-
\chapter{Architecture}

You can find some info about clang internal architecture and relation
with other LLVM components.

When we spoke about \textbf{clang} we should separate 2 things:
\begin{itemize}
\item driver
\item compiler frontend 
\end{itemize}
Both of them are called \textbf{clang} but perform different
operations. The driver invokes different stages of compilation process
(see fig. \ref{fig:clang_driver}). 
\begin{figure}
\begin{center}
\smartdiagramset{border color=none,
back arrow disabled=true}
\smartdiagram[flow diagram:horizontal]{Frontend,Middle-end, Backend,Assembler,Linker}
\end{center}
  \caption{Clang driver}
  \label{fig:clang_driver}
\end{figure}
The stages are standard for ordinary compiler and nothing special is
there:
\begin{itemize}
\item Frontend: it does lexical analysis and parsing.
\item Middle-end: it does different optimization on the intermediate
  representation (LLVM-IR) code
\item Backend: Native code generation
\item Assembler: Running assembler
\item Linker: Running linker
\end{itemize}

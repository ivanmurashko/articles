%% -*- coding:utf-8 -*-
\chapter{Features}

You can find some info about different clang features at the chapter

\section{Precompiled headers}
Precompiled headers or \textbf{pch} is a clang feature that was
designed with the goal to improve clang frontend performance. The
basic idea was to create AST for a header file and reuse the AST for
some purposes.

\subsection{User guide}
Generate you pch file is simple \cite{clang:user_manual}. Suppose you have a header file with
name \textbf{header.h}:
\inputminted{c++}{./src/pch/simple/header.h} then you can generate a pch for it with
\begin{verbatim}
clang -x c++-header header.h -o header.pch
\end{verbatim}
the option \textbf{-x c++-header} was used there. The option says that
the header file has to be treated as a c++ header file. The output
file is \textbf{header.pch}.

The precompiled headers generation is not enough and you may want to
start using them. Typical C++ source file that uses the header may
look like
\inputminted{c++}{./src/pch/simple/main.cpp}
As you may see, the header is included as follows
\begin{minted}{c++}
  ...
#include "header.h"
  ...
\end{minted}
By default clang will not use a pch at the case and you have to
specify it explicitly with
\begin{verbatim}
clang -include-pch header.pch main.cpp -o main -lstdc++
\end{verbatim}
We can check the command with debugger and it will give us
\begin{minted}[breaklines]{text}
$ lldb ~/local/llvm-project/build/bin/clang -- -cc1 -include-pch header.pch main.cpp -fsyntax-only
...
(lldb) b clang::ASTReader::ReadAST                                             
...
(lldb) r
...
   4231   llvm::SaveAndRestore<SourceLocation>
-> 4232     SetCurImportLocRAII(CurrentImportLoc, ImportLoc);
   4233   llvm::SaveAndRestore<Optional<ModuleKind>> SetCurModuleKindRAII(
   4234       CurrentDeserializingModuleKind, Type);
   4235 
(lldb) p FileName
(llvm::StringRef) $0 = (Data = "header.pch", Length = 10)
\end{minted}
Note that only the first \textbf{--include-pch} option will be
processed, all others will be ignored. It reflects the fact that there
can be only one precompiled header for a translation unit.

\section{Modules}
TBD

